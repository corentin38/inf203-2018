\documentclass[10pt]{article}

\usepackage{graphics}
\usepackage{dirtree}
\usepackage{paracol}
\usepackage{epigraph}
\usepackage{enumitem}
\usepackage{xcolor}
\usepackage{fancyvrb}
\usepackage{calc}

\setlength{\textwidth}{17cm}
\setlength{\oddsidemargin}{-1cm}
\setlength{\evensidemargin}{-1cm}
\setlength{\textheight}{26cm}
\setlength{\parindent}{0pt}
\setlength{\parskip}{0.3ex}

\usepackage{fancyhdr}
\usepackage{epsfig}
\usepackage[utf8]{inputenc}
\usepackage[french]{babel}
\usepackage[T1]{fontenc}
%\usepackage{verbatim}
\usepackage{graphics}
\usepackage{amsmath,amsfonts,amssymb}
\usepackage{listings}
\usepackage{thmbox}
\usepackage{comment}

\oddsidemargin=0cm
\evensidemargin=0cm
\textwidth=17cm
\textheight=25cm
\topmargin=0mm
\hoffset=-6mm
\voffset=-25mm
%\headsep=30pt

\newcounter{numpartie}
\newcounter{exo}
\newcommand{\partie}[2]
{
        \refstepcounter{numpartie}
        \setcounter{exo}{0}
        \vspace{3mm}

        %\noindent{\large\bf Partie \Alph{numpartie}~-~#1}\hfill
        \noindent{\large\bf Partie \Alph{numpartie}~ ~#1}\hfill
        %[\;{\it bar\^eme indicatif : #2}\;]\\
        \\

}

\newenvironment{exercice}{\refstepcounter{exo} \vspace*{1em}\begin{thmbox}[S]{{\bf Exercice \arabic{exo}~:}}}{\end{thmbox}}

\newenvironment{solution}{\vspace*{1em}\begin{thmbox}[S]{{\bf Solution~:}}}{\end{thmbox}}

\newcommand{\IGNORE}[1]{}

\def\EnteteUE{
\noindent{\bf Université Grenoble Alpes} \hfill {\bf DLST}

\vspace{-8pt}

\noindent\hrulefill

\vspace{-2pt}

\noindent{\bf UE INF203} \hfill {\bf Année 2016-17}

\vspace*{.5cm}
}

\newcommand{\PiedDePageUE}[1]{
\fancypagestyle{monstyle}
{
\rhead{}
\lfoot{{\bf\small INF203 - 2016/2017}}
\cfoot{\thepage}
\rfoot{{\it #1}}
\renewcommand{\footrulewidth}{0.5pt}
\renewcommand{\headrulewidth}{0.0pt}
}
\pagestyle{monstyle}
}


\newcounter{question}
\newenvironment{question}[1][]{\refstepcounter{question}
   \textbf{[\bf \alph{question}] {\begin{emph}{#1}\end{emph}}}}{~$\blacksquare$~\\}

\newcommand{\unix}[1]{\hspace*{2cm}{\bf \tt #1}}
\newcommand{\fich}[1]{{\bf \em #1}}
\newcommand{\ecran}[1]{{\tt #1}}
\newcommand{\C}[1]{{\tt #1}}
\newcommand{\cmd}[1]{{\bf #1}}
\newcommand{\home}{\~{}}

\newenvironment{exocomp}{\vspace*{1em}\begin{thmbox}[M]{Exercice complémentaire :~}}{\end{thmbox}}


\begin{document}
\thispagestyle{empty}

\definecolor{mygreen}{rgb}{0,0.6,0}
\definecolor{mygray}{rgb}{0.5,0.5,0.5}
\definecolor{mymauve}{rgb}{0.58,0,0.82}

\lstdefinestyle{customc}{
  belowcaptionskip=1\baselineskip,
  breaklines=true,
  frame=L,
  xleftmargin=\parindent,
  language=C,
  showstringspaces=false,
  basicstyle=\footnotesize\ttfamily,
  keywordstyle=\bfseries\color{green!40!black},
  commentstyle=\itshape\color{purple!40!black},
  identifierstyle=\color{blue},
  stringstyle=\color{orange},
}

\lstdefinestyle{none}{
  belowcaptionskip=1\baselineskip,
  breaklines=true,
  frame=L,
  xleftmargin=\parindent,
  language=bash,
  showstringspaces=false,
  basicstyle=\footnotesize\ttfamily,
  keywordstyle=\bfseries\color{black},
  commentstyle=\itshape\color{black},
  identifierstyle=\color{black},
  stringstyle=\color{black},
}

\lstset{tabsize=3, style=customc,literate=
  {á}{{\'a}}1 {é}{{\'e}}1 {í}{{\'i}}1 {ó}{{\'o}}1 {ú}{{\'u}}1
  {Á}{{\'A}}1 {É}{{\'E}}1 {Í}{{\'I}}1 {Ó}{{\'O}}1 {Ú}{{\'U}}1
  {à}{{\`a}}1 {è}{{\`e}}1 {ì}{{\`i}}1 {ò}{{\`o}}1 {ù}{{\`u}}1
  {À}{{\`A}}1 {È}{{\'E}}1 {Ì}{{\`I}}1 {Ò}{{\`O}}1 {Ù}{{\`U}}1
  {ä}{{\"a}}1 {ë}{{\"e}}1 {ï}{{\"i}}1 {ö}{{\"o}}1 {ü}{{\"u}}1
  {Ä}{{\"A}}1 {Ë}{{\"E}}1 {Ï}{{\"I}}1 {Ö}{{\"O}}1 {Ü}{{\"U}}1
  {â}{{\^a}}1 {ê}{{\^e}}1 {î}{{\^i}}1 {ô}{{\^o}}1 {û}{{\^u}}1
  {Â}{{\^A}}1 {Ê}{{\^E}}1 {Î}{{\^I}}1 {Ô}{{\^O}}1 {Û}{{\^U}}1
  {œ}{{\oe}}1 {Œ}{{\OE}}1 {æ}{{\ae}}1 {Æ}{{\AE}}1 {ß}{{\ss}}1
  {ű}{{\H{u}}}1 {Ű}{{\H{U}}}1 {ő}{{\H{o}}}1 {Ő}{{\H{O}}}1
  {ç}{{\c c}}1 {Ç}{{\c C}}1 {ø}{{\o}}1 {å}{{\r a}}1 {Å}{{\r A}}1
  {€}{{\euro}}1 {£}{{\pounds}}1 {«}{{\guillemotleft}}1
  {»}{{\guillemotright}}1 {ñ}{{\~n}}1 {Ñ}{{\~N}}1 {¿}{{?`}}1
}

\EnteteUE

\begin{center}
  {\large {\bf Corrigé du DS de mi-semestre}}
\end{center}

\vspace*{2cm}

\begin{center}
  {\large {\bf Unix}}
\end{center}

\section{Beaucoup de programmes}

\begin{enumerate}
  \setlength\itemsep{1em}

\item Créer un répertoire \texttt{Programmes} et y copier tous nos fichiers C :

  \lstset{style=none}
  \begin{lstlisting}
    mkdir Programmes
    cp INF203/TP*/*.c Programmes
  \end{lstlisting}


\item Donner accès au répertoire \texttt{Programmes} à votre binôme :

  \lstset{style=none}
  \begin{lstlisting}
    chmod g+x ~                  # Donner au binôme le droit de traverser ~
    chmod g+x Programmes         # Donner au binôme le droit de traverser Programmes
  \end{lstlisting}

  \vspace{0.5cm}

  \textbf{Petites précisions :} \\
  Après les commandes ci-dessus, l'utilisateur a le droit de
  \textit{traverser} notre répertoire de login, mais pas de le lire. Cela
  signifie qu'il pourra faire un \texttt{cd <mon\_login>} mais qu'une
  fois à l'intérieur, il ne pourra pas faire un \texttt{ls}. Il lui
  faudra connaître le nom exact du répertoire \texttt{Programmes} pour
  faire la commande \texttt{cd~Programmes}. Ce dernier lui donne par
  contre les droits \texttt{r} et \texttt{x}, donc non-seulement celui
  de faire un \texttt{cd} dedans, mais aussi celui de faire un
  \texttt{ls}. Par la suite, tous les programmes ont déjà le droit
  \texttt{r} pour les utilisateurs du même groupe que nous, le binôme
  pourra donc faire la commande \texttt{cp} pour les recopier chez lui
  sans problème.

\item Pour déplacer \texttt{Programmes} dans \texttt{Archives} et le
  renommer :

  \lstset{style=none}
  \begin{lstlisting}
    mv Programmes Archives/Programmes_2016-2017
  \end{lstlisting}

\item Pour supprimer tous les fichier / répertoires du dossier
  \texttt{Archives} dont le nom contient \texttt{2016-2017} :

  \lstset{style=none}
  \begin{lstlisting}
    rm -r Archives/*2016-2017*
  \end{lstlisting}

\item Pour compter le nombre de fichiers ayant 203 et 42 dans leur nom
  dans tous nos programmes en C, pas le choix, il faut utiliser la
  commande \texttt{find} et se souvenir de sa syntaxe exotique.

  \lstset{style=none}
  \begin{lstlisting}
    find INF203 \( -name "*.c" -a -name "*42*" -a -name "*203*" \) | wc -l
  \end{lstlisting}

  \vspace{0.5cm}
  Explications:
  \vspace{0.2cm}

  \begin{tabular}{l p{10cm}}
    \verb|\( expr1 -a expr2 \)| & permet de combiner les
                                  expressions. Ici, \texttt{find} ne listera
                                  que les fichiers respectant
                                  \texttt{expr1} \textbf{et}
                                  \texttt{expr2}. \vspace{0.2cm} \\
    \verb|-name "*42*"|         & est l'expression pour indiquer que le nom doit
                                  contenir 42. (Pour les connaisseurs,
                                  attention, ce n'est pas une
                                  regex, mais un pattern shell) \vspace{0.2cm} \\
    \verb#| wc -l#              & à la fin de la commande redirige la sortie de
                                  \texttt{find} vers \texttt{wc} qui, avec l'option
                                  \texttt{-l}, compte les lignes \\
  \end{tabular}

\end{enumerate}

%\begin{enumerate}[label=\textbf{[\alph*]},resume]
%  \setlength\itemsep{1em}
%
%%  \lstinputlisting{listing/billes.c}
%
%\begin{Verbatim}[xleftmargin=2em]
%corentin@gazelle:listing $ diff d1.c d2.c
%corentin@gazelle:listing $
%\end{Verbatim}
%
%\end{enumerate}

\end{document}
