\documentclass[10pt]{article}

\usepackage{graphics}
\usepackage{dirtree}
\usepackage{paracol}
\usepackage{epigraph}
\usepackage{enumitem}
\usepackage{xcolor}

\setlength{\textwidth}{17cm}
\setlength{\oddsidemargin}{-1cm}
\setlength{\evensidemargin}{-1cm}
\setlength{\textheight}{26cm}
\setlength{\parindent}{0pt}
\setlength{\parskip}{0.3ex}

\usepackage{fancyhdr}
\usepackage{epsfig}
\usepackage[utf8]{inputenc}
\usepackage[french]{babel}
\usepackage[T1]{fontenc}
%\usepackage{verbatim}
\usepackage{graphics}
\usepackage{amsmath,amsfonts,amssymb}
\usepackage{listings}
\usepackage{thmbox}
\usepackage{comment}

\oddsidemargin=0cm
\evensidemargin=0cm
\textwidth=17cm
\textheight=25cm
\topmargin=0mm
\hoffset=-6mm
\voffset=-25mm
%\headsep=30pt

\newcounter{numpartie}
\newcounter{exo}
\newcommand{\partie}[2]
{
        \refstepcounter{numpartie}
        \setcounter{exo}{0}
        \vspace{3mm}

        %\noindent{\large\bf Partie \Alph{numpartie}~-~#1}\hfill
        \noindent{\large\bf Partie \Alph{numpartie}~ ~#1}\hfill
        %[\;{\it bar\^eme indicatif : #2}\;]\\
        \\

}

\newenvironment{exercice}{\refstepcounter{exo} \vspace*{1em}\begin{thmbox}[S]{{\bf Exercice \arabic{exo}~:}}}{\end{thmbox}}

\newenvironment{solution}{\vspace*{1em}\begin{thmbox}[S]{{\bf Solution~:}}}{\end{thmbox}}

\newcommand{\IGNORE}[1]{}

\def\EnteteUE{
\noindent{\bf Université Grenoble Alpes} \hfill {\bf DLST}

\vspace{-8pt}

\noindent\hrulefill

\vspace{-2pt}

\noindent{\bf UE INF203} \hfill {\bf Année 2017-18}

\vspace*{.5cm}
}

\newcommand{\PiedDePageUE}[1]{
\fancypagestyle{monstyle}
{
\rhead{}
\lfoot{{\bf\small INF203 - 2016/2017}}
\cfoot{\thepage}
\rfoot{{\it #1}}
\renewcommand{\footrulewidth}{0.5pt}
\renewcommand{\headrulewidth}{0.0pt}
}
\pagestyle{monstyle}
}


\newcounter{question}
\newenvironment{question}[1][]{\refstepcounter{question}
   \textbf{[\bf \alph{question}] {\begin{emph}{#1}\end{emph}}}}{~$\blacksquare$~\\}

\newcommand{\unix}[1]{\hspace*{2cm}{\bf \tt #1}}
\newcommand{\fich}[1]{{\bf \em #1}}
\newcommand{\ecran}[1]{{\tt #1}}
\newcommand{\C}[1]{{\tt #1}}
\newcommand{\cmd}[1]{{\bf #1}}
\newcommand{\home}{\~{}}

\newenvironment{exocomp}{\vspace*{1em}\begin{thmbox}[M]{Exercice complémentaire :~}}{\end{thmbox}}


\begin{document}
\thispagestyle{empty}

\definecolor{mygreen}{rgb}{0,0.6,0}
\definecolor{mygray}{rgb}{0.5,0.5,0.5}
\definecolor{mymauve}{rgb}{0.58,0,0.82}

\lstdefinestyle{customc}{
  belowcaptionskip=1\baselineskip,
  breaklines=true,
  frame=L,
  xleftmargin=\parindent,
  language=C,
  showstringspaces=false,
  basicstyle=\footnotesize\ttfamily,
  keywordstyle=\bfseries\color{green!40!black},
  commentstyle=\itshape\color{purple!40!black},
  identifierstyle=\color{blue},
  stringstyle=\color{orange},
}

\lstdefinestyle{none}{
  belowcaptionskip=1\baselineskip,
  breaklines=true,
  frame=L,
  xleftmargin=\parindent,
  language=bash,
  showstringspaces=false,
  basicstyle=\footnotesize\ttfamily,
  keywordstyle=\bfseries\color{black},
  commentstyle=\itshape\color{black},
  identifierstyle=\color{black},
  stringstyle=\color{black},
}

\lstset{tabsize=3, style=customc,literate=
  {á}{{\'a}}1 {é}{{\'e}}1 {í}{{\'i}}1 {ó}{{\'o}}1 {ú}{{\'u}}1
  {Á}{{\'A}}1 {É}{{\'E}}1 {Í}{{\'I}}1 {Ó}{{\'O}}1 {Ú}{{\'U}}1
  {à}{{\`a}}1 {è}{{\`e}}1 {ì}{{\`i}}1 {ò}{{\`o}}1 {ù}{{\`u}}1
  {À}{{\`A}}1 {È}{{\'E}}1 {Ì}{{\`I}}1 {Ò}{{\`O}}1 {Ù}{{\`U}}1
  {ä}{{\"a}}1 {ë}{{\"e}}1 {ï}{{\"i}}1 {ö}{{\"o}}1 {ü}{{\"u}}1
  {Ä}{{\"A}}1 {Ë}{{\"E}}1 {Ï}{{\"I}}1 {Ö}{{\"O}}1 {Ü}{{\"U}}1
  {â}{{\^a}}1 {ê}{{\^e}}1 {î}{{\^i}}1 {ô}{{\^o}}1 {û}{{\^u}}1
  {Â}{{\^A}}1 {Ê}{{\^E}}1 {Î}{{\^I}}1 {Ô}{{\^O}}1 {Û}{{\^U}}1
  {œ}{{\oe}}1 {Œ}{{\OE}}1 {æ}{{\ae}}1 {Æ}{{\AE}}1 {ß}{{\ss}}1
  {ű}{{\H{u}}}1 {Ű}{{\H{U}}}1 {ő}{{\H{o}}}1 {Ő}{{\H{O}}}1
  {ç}{{\c c}}1 {Ç}{{\c C}}1 {ø}{{\o}}1 {å}{{\r a}}1 {Å}{{\r A}}1
  {€}{{\euro}}1 {£}{{\pounds}}1 {«}{{\guillemotleft}}1
  {»}{{\guillemotright}}1 {ñ}{{\~n}}1 {Ñ}{{\~N}}1 {¿}{{?`}}1
}

\EnteteUE

\begin{center}
  {\large {\bf Corrigé TP2}}
\end{center}

\vspace*{0.5cm}

\section{{\tt installeTP.sh}}

Normalement, à l'issue du premier TP, vous devez avoir dans votre
répertoire personnel ({\tt \~/INF203/TP1/scripts}) le script {\tt installeTP.sh}.
Vous pouvez le copier directement dans votre dossier personnel ({\tt \~/installeTP.sh})
car il vous servira désormais à chaque début de TP à récupérer les
fichiers dont vous aurez besoin. Ainsi, au début du TP2, vous pouvez
taper la commande suivante :

\begin{verbatim}
    ./installeTP.sh 2
\end{verbatim}

Où le {\tt 2} désigne le TP2. Cela remplace avantageusement la
commande que vous aviez dû taper la semaine dernière pour récupérer
les fichiers du TP1, à savoir :

\begin{verbatim}
    cp -R /Public/203_INF_Public/TP1 INF203          # Ne plus jamais faire ça !
\end{verbatim}

\section{Compilation}

Pour commencer le TP2, vous devez être au clair avec la
\textbf{compilation}. La compilation, c'est l'étape où votre programme
qui se trouve dans un fichier {\tt .c} (comme par exemple {\tt deborde\_char.c})
va être traduit en instructions compréhensibles par l'ordinateur.

\vspace*{0.2cm}

Avant :

\lstinputlisting{listing/ex.c}

Après :

\begin{lstlisting}
7f 45 4c 46 02 01 01 00  00 00 00 00 00 00 00 00
03 00 3e 00 01 00 00 00  80 05 00 00 00 00 00 00
40 00 00 00 00 00 00 00  38 22 00 00 00 00 00 00
00 00 00 00 40 00 38 00  09 00 40 00 24 00 23 00
06 00 00 00 05 00 00 00  40 00 00 00 00 00 00 00
\end{lstlisting}

Pour cela, on utilise la commande {\tt clang} (Pour les anciens, il
s'agit d'un équivalent à la commande {\tt gcc}). Pour compiler le
fichier {\tt monprog.c}, par exemple, on exécutera donc la commande :

\begin{verbatim}
    clang monprog.c
\end{verbatim}

Cela créera un exécutable nommé {\tt a.out}. On peut aussi entrer la
commande suivante :

\begin{verbatim}
    clang monprog.c -o mon_executable
\end{verbatim}

Cela compilera de la même manière, mais en nommant l'exécutable
produit : {\tt mon\_executable}. On pourra ensuite lancer notre
programme en entrant la commande :

\begin{verbatim}
    ./mon_executable
\end{verbatim}

On peut maintenant commener à répondre aux questions ! Ouf !

\section{Testons des entiers}

\begin{enumerate}[label=\textbf{[\alph*]}]
  \setlength\itemsep{1em}

\item Pour ne pas retaper à chaque fois les mêmes commandes, on peut
  utiliser les flèches du haut et du bas pour remonter dans
  l'historique des commandes.

  On peut aussi utiliser la touche TAB pour compéter dès que possible
  les noms de fichiers ainsi que les commandes que l'on est en train
  de taper.

\item On pars du programme {\tt exemple\_generation.c} que l'on a un
  peu modifié pour qu'il affiche ``Trop grand'', ``Trop petit'' ou
  ``Youpie !'' :

  \lstinputlisting{listing/exemple_generation_1.c}

  Une fois compilé, si on lance le programme plusieurs fois, on voit
  que l'on tombe souvent sur ``Trop grand'' et ``Trop petit'', mais
  rarement sur ``Youpie !''. On souhaite donc que le programme fasse
  lui même un certain nombre de générations. On utilise pour cela une
  boucle.

  \lstinputlisting{listing/exemple_generation_2.c}

  \textbf{NB:} La boucle {\tt for} part de {\tt i = 0} et s'arrête
  quand {\tt i} n'est plus {\tt < nombre\_boucle}.

  Donc {\tt i} va de {\tt 0} à {\tt nombre\_boucle - 1}.

\end{enumerate}

\newpage

\section{Provoquons un débordement}

\begin{enumerate}[label=\textbf{[\alph*]},resume]
  \setlength\itemsep{1em}

\item \textbf{[d] [e]} Pour les trois précédentes questions, voilà le tableau résumé :

  \begin{center}
    \begin{tabular}{|l|l|l|l|l|}
      \hline
      Type & Taille (octet) & Taille (bit) & Valeur Max & Temps débordement \\
      \hline
      \textbf{unsigned char}  & 1 & 8  & 0 .. 255        & < 0.005 s \\
      \textbf{unsigned short} & 2 & 16 & 0 .. 65335      & < 0.005 s \\
      \textbf{unsigned int}   & 4 & 32 & 0 .. 4294967295 & env. 10 s  \\
      \hline
    \end{tabular}
  \end{center}

\setcounter{enumi}{5}
\item Comme l'ordinateur compte à la même vitesse quelque soit le type
  de la variable, on s'attend à ce que {\tt deborde\_int} prenne
  $\frac{{\tt taille(int)}}{{\tt taille(short)}}$ fois plus de temps que {\tt
  deborde\_short}, soit 65536 fois plus de temps. Cela dit, un temps
  d'exécution inférieur à 0.005 secondes n'est pas significatif. On ne
  peut donc pas vraiment déduire quoi que ce soit des mesures
  précédentes ...

\end{enumerate}

\section{Rangeons !}

\begin{enumerate}[label=\textbf{[\alph*]},resume]
  \setlength\itemsep{1em}

\item Appel effectif à la fonction {\tt inserer} :

\begin{verbatim}
    inserer(T, i, valeur);
\end{verbatim}

\item Listing complet de {\tt exemple\_generation\_tableau.c} :

  \lstinputlisting{listing/exemple_generation_tableau.c}

\end{enumerate}

\newpage

\section{Une (petite) commande de la semaine : {\tt wc}}

\begin{enumerate}[label=\textbf{[\alph*]},resume]
  \setlength\itemsep{1em}

\item La commande {\tt wc} pour \textit{word count}, permet de compter
  les lignes d'un fichier avec l'option \textbf{{\tt -l}}. Exemple :

\begin{verbatim}
    wc -l fichier.c
\end{verbatim}

On note que {\tt wc} ne fait que compter le nombre d'occurence du
caractère de retour à la ligne ``{\tt \textbackslash n}''. Donc si la dernière ligne
n'est pas suivie d'un retour à la ligne, elle n'est pas comptée.

\item Pour compter toutes les lignes de code C que l'on a produites,
  on peut utiliser la commande suivante :

\begin{verbatim}
    wc -l *.c
\end{verbatim}

\end{enumerate}

\section{Installons le TP}

Revenons sur les droits d'accès. On sait qu'il peut y avoir
plusieurs utilisateurs sur un système UNIX comme Turing. En
l'occurence, chaque étudiant a un compte. Il peut y avoir également
plusieurs groupes. Chaque utilisateur peut appartenir ou non à chacun
des groupes. On peut lister les groupes auxquels on appartient en
tapant la commande :

\begin{verbatim}
    $ groups
    corentin cdrom floppy sudo audio dip video plugdev netdev
    bluetooth lpadmin scanner
\end{verbatim}

Dans cet exemple, on voit que sur mon ordinateur personnel,
j'appartiens entre autres au groupe {\tt cdrom}, ce qui me permet
d'accéder au lecteur de CD, mais aussi au groupe {\tt audio}, ce qui
me permet d'écouter de la musique et d'enregistrer des sons.

Sous UNIX, chaque fichier appartient à UN utilisateur et à UN
groupe. Par exemple pour le fichier suivant :

\begin{verbatim}
    $ ls -l logo.jpg
    -rw-r--r-- 1 corentin users 275085 avril 19  2017 logo.jpg
\end{verbatim}

On voit que le fichier appartient à l'utilisateur {\tt corentin} et au
groupe {\tt users}. On voit également que les droits sont définis de
la manière suivante :

\begin{verbatim}
    rw-r--r--
\end{verbatim}

Soit :

\begin{center}
  \begin{tabular}{|c|c|c|c|c|c|c|c|c|}
    \hline
    \multicolumn{3}{|c|}{u} & \multicolumn{3}{c|}{g} & \multicolumn{3}{c|}{o} \\
    \hline
    r & w & x & r & w & x & r & w & x \\
    \hline
    {\tt X} & {\tt X} &  & {\tt X} &  &  & {\tt X} &  &  \\
    \hline
  \end{tabular}
\end{center}

Donc, l'utilisateur {\tt corentin} peut lire et modifier le fichier et
les membres du groupe {\tt users} et le reste des utilisateurs du
système (donc tous ceux qui ne sont pas dans le groupe {\tt users}) ne
peuvent que le lire.


\begin{enumerate}[label=\textbf{[\alph*]},resume]
  \setlength\itemsep{1em}

\item Dans cet exemple, l'utilisateur {\tt NOM\_A} essaye de copier le
  fichier {\tt installeTP.sh} qui se trouve dans son répertoire
  personnel vers le répertoire personnel de {\tt NOM\_B}. Ce n'est pas
  possible car seul {\tt NOM\_B} a les droits nécessaires pour écrire
  dans son répertoire personnel. Les autres utilisateurs peuvent voir
  ses fichiers, les lire, mais pas les modifier. Vous pouvez vérifier
  cela en tapant la commande {\tt ls -l} dans votre répertoire
  personnel. Vous verrez que tous vos fichiers et répertoires ont pour
  propriétaire vous-même et que leur droits interdisent la
  modification pour tous les autres utilisateurs (pas de droit {\tt
    w}).

\item Pour récupérer le fichier {\tt installeTP.sh} sur le compte de
  votre binôme, on est donc obligé de procéder à l'inverse. On se
  connecte en tant que {\tt NOM\_B} et on copie {\tt installeTP.sh}
  depuis le répertoire de {\tt NOM\_A}. On a le droit d'effectuer
  cette copie car on dispose des droits en lecture sur les fichiers de
  {\tt NOM\_A} (droit {\tt r}) et que la copie ne modifie pas le
  fichier original. On entre donc la commande :

\begin{verbatim}
    cp /home/n/NOM_A/installeTP.sh ~
\end{verbatim}

%\vspace*{0.5cm}
%
%
%\columnratio{0.8,0.2}
%\begin{paracol}{2}
%  \begin{leftcolumn}
%    \begin{enumerate}
%      \setcounter{enumi}{1}
%    \item Edgar se place dans son répertoire personnel et lance la commande :
%\begin{verbatim}
%        cp ../dude/../dude/tp2 ~/./révision/tp2
%\end{verbatim}
%Sa commande va-t-elle fonctionner ? Comment aurait-il pu la simplifier ?
%    \end{enumerate}
%  \end{leftcolumn}
%
%\begin{rightcolumn}
%  \dirtree{%
%    .1 /.
%    .2 home.
%    .3 dude.
%    .4 tp2.
%    .3 edgar.
%    .4 révision.
%  }
%\end{rightcolumn}
%\end{paracol}
%
%\vspace{2cm}
%
%\begin{enumerate}
%  \setcounter{enumi}{2}
%\item On souhaite écrire un script qui indique l'appartenance d'un entier à un intervalle. Par exemple, on doit pouvoir vérifier que 4 est compris entre 3 et 10 en tapant :
%\begin{verbatim}
%  ./intervalle.sh 3 10 4
%\end{verbatim}
%\begin{enumerate}
%\item Écrire le test (le if) pour vérifier que l'utilisateur a bien donné 3 arguments
%
%  \vspace{2cm}
%
%\item Écrire le bout de code qui affiche {\tt OK} si le 3\textsuperscript{ème} argument est bien compris entre le 1\textsuperscript{er} et le 2\textsuperscript{ème}, et {\tt KO} sinon
%\end{enumerate}
%
%\newpage
%
%\item Écrire un script qui liste tous les fichiers finissant par {\tt .sh} et qui leur donne le droit d'exécution si :
%  \begin{itemize}
%  \item Le fichier n'est pas déjà exécutable
%  \item Le fichier n'est pas un répertoire
%  \item Le fichier n'est pas vide
%  \end{itemize}
%
%\vspace{8cm}
%
%\item Edgar (encore lui) souhaite avoir un programme qui affiche un compte à rebours (10, 9, 8 ...). Pour ce faire, il écrit le script suivant :
%
%  \vspace{0.2cm}
%
%\begin{verbatim}
%1       #!/bin/bash
%2       decompte() {
%3           if [ $1 -eq 0 ]      # Si on arrive à 0, on s'arrête
%4           then
%5               exit 0
%6           fi
%7           echo $1              # On dit où on en est
%8           suiv=$(expr $1 - 1)  # On calcule le chiffre suivant
%9           decompte $suiv       # On appelle la fonction avec le chiffre suivant
%10      }
%11
%12      decompte 10              # On lance un décompte à partir de 10
%\end{verbatim}
%
%  \vspace{0.2cm}
%
%\begin{enumerate}
%\item Est-ce que son script va fonctionner ?
%
%  \vspace{2cm}
%
%\item Si, à la ligne 9, Edgar avait appelé : {\tt decompte \$1} au lieu de : {\tt decompte \$suiv}, que se serait-il passé ?
%\end{enumerate}
%
%
%\end{enumerate}
%
%\vspace{2cm}
  %
\end{enumerate}
\end{document}
