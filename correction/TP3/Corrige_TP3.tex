\documentclass[10pt]{article}

\usepackage{graphics}
\usepackage{dirtree}
\usepackage{paracol}
\usepackage{epigraph}
\usepackage{enumitem}
\usepackage{xcolor}
\usepackage{fancyvrb}
\usepackage{calc}

\setlength{\textwidth}{17cm}
\setlength{\oddsidemargin}{-1cm}
\setlength{\evensidemargin}{-1cm}
\setlength{\textheight}{26cm}
\setlength{\parindent}{0pt}
\setlength{\parskip}{0.3ex}

\usepackage{fancyhdr}
\usepackage{epsfig}
\usepackage[utf8]{inputenc}
\usepackage[french]{babel}
\usepackage[T1]{fontenc}
%\usepackage{verbatim}
\usepackage{graphics}
\usepackage{amsmath,amsfonts,amssymb}
\usepackage{listings}
\usepackage{thmbox}
\usepackage{comment}

\oddsidemargin=0cm
\evensidemargin=0cm
\textwidth=17cm
\textheight=25cm
\topmargin=0mm
\hoffset=-6mm
\voffset=-25mm
%\headsep=30pt

\newcounter{numpartie}
\newcounter{exo}
\newcommand{\partie}[2]
{
        \refstepcounter{numpartie}
        \setcounter{exo}{0}
        \vspace{3mm}

        %\noindent{\large\bf Partie \Alph{numpartie}~-~#1}\hfill
        \noindent{\large\bf Partie \Alph{numpartie}~ ~#1}\hfill
        %[\;{\it bar\^eme indicatif : #2}\;]\\
        \\

}

\newenvironment{exercice}{\refstepcounter{exo} \vspace*{1em}\begin{thmbox}[S]{{\bf Exercice \arabic{exo}~:}}}{\end{thmbox}}

\newenvironment{solution}{\vspace*{1em}\begin{thmbox}[S]{{\bf Solution~:}}}{\end{thmbox}}

\newcommand{\IGNORE}[1]{}

\def\EnteteUE{
\noindent{\bf Université Grenoble Alpes} \hfill {\bf DLST}

\vspace{-8pt}

\noindent\hrulefill

\vspace{-2pt}

\noindent{\bf UE INF203} \hfill {\bf Année 2016-17}

\vspace*{.5cm}
}

\newcommand{\PiedDePageUE}[1]{
\fancypagestyle{monstyle}
{
\rhead{}
\lfoot{{\bf\small INF203 - 2016/2017}}
\cfoot{\thepage}
\rfoot{{\it #1}}
\renewcommand{\footrulewidth}{0.5pt}
\renewcommand{\headrulewidth}{0.0pt}
}
\pagestyle{monstyle}
}


\newcounter{question}
\newenvironment{question}[1][]{\refstepcounter{question}
   \textbf{[\bf \alph{question}] {\begin{emph}{#1}\end{emph}}}}{~$\blacksquare$~\\}

\newcommand{\unix}[1]{\hspace*{2cm}{\bf \tt #1}}
\newcommand{\fich}[1]{{\bf \em #1}}
\newcommand{\ecran}[1]{{\tt #1}}
\newcommand{\C}[1]{{\tt #1}}
\newcommand{\cmd}[1]{{\bf #1}}
\newcommand{\home}{\~{}}

\newenvironment{exocomp}{\vspace*{1em}\begin{thmbox}[M]{Exercice complémentaire :~}}{\end{thmbox}}


\begin{document}
\thispagestyle{empty}

\definecolor{mygreen}{rgb}{0,0.6,0}
\definecolor{mygray}{rgb}{0.5,0.5,0.5}
\definecolor{mymauve}{rgb}{0.58,0,0.82}

\lstdefinestyle{customc}{
  belowcaptionskip=1\baselineskip,
  breaklines=true,
  frame=L,
  xleftmargin=\parindent,
  language=C,
  showstringspaces=false,
  basicstyle=\footnotesize\ttfamily,
  keywordstyle=\bfseries\color{green!40!black},
  commentstyle=\itshape\color{purple!40!black},
  identifierstyle=\color{blue},
  stringstyle=\color{orange},
}

\lstdefinestyle{none}{
  belowcaptionskip=1\baselineskip,
  breaklines=true,
  frame=L,
  xleftmargin=\parindent,
  language=bash,
  showstringspaces=false,
  basicstyle=\footnotesize\ttfamily,
  keywordstyle=\bfseries\color{black},
  commentstyle=\itshape\color{black},
  identifierstyle=\color{black},
  stringstyle=\color{black},
}

\lstset{tabsize=3, style=customc,literate=
  {á}{{\'a}}1 {é}{{\'e}}1 {í}{{\'i}}1 {ó}{{\'o}}1 {ú}{{\'u}}1
  {Á}{{\'A}}1 {É}{{\'E}}1 {Í}{{\'I}}1 {Ó}{{\'O}}1 {Ú}{{\'U}}1
  {à}{{\`a}}1 {è}{{\`e}}1 {ì}{{\`i}}1 {ò}{{\`o}}1 {ù}{{\`u}}1
  {À}{{\`A}}1 {È}{{\'E}}1 {Ì}{{\`I}}1 {Ò}{{\`O}}1 {Ù}{{\`U}}1
  {ä}{{\"a}}1 {ë}{{\"e}}1 {ï}{{\"i}}1 {ö}{{\"o}}1 {ü}{{\"u}}1
  {Ä}{{\"A}}1 {Ë}{{\"E}}1 {Ï}{{\"I}}1 {Ö}{{\"O}}1 {Ü}{{\"U}}1
  {â}{{\^a}}1 {ê}{{\^e}}1 {î}{{\^i}}1 {ô}{{\^o}}1 {û}{{\^u}}1
  {Â}{{\^A}}1 {Ê}{{\^E}}1 {Î}{{\^I}}1 {Ô}{{\^O}}1 {Û}{{\^U}}1
  {œ}{{\oe}}1 {Œ}{{\OE}}1 {æ}{{\ae}}1 {Æ}{{\AE}}1 {ß}{{\ss}}1
  {ű}{{\H{u}}}1 {Ű}{{\H{U}}}1 {ő}{{\H{o}}}1 {Ő}{{\H{O}}}1
  {ç}{{\c c}}1 {Ç}{{\c C}}1 {ø}{{\o}}1 {å}{{\r a}}1 {Å}{{\r A}}1
  {€}{{\euro}}1 {£}{{\pounds}}1 {«}{{\guillemotleft}}1
  {»}{{\guillemotright}}1 {ñ}{{\~n}}1 {Ñ}{{\~N}}1 {¿}{{?`}}1
}

\EnteteUE

\begin{center}
  {\large {\bf Corrigé TP3}}
\end{center}

\vspace*{0.5cm}

\section{INF203}

Se référer au corrigé du TP2 pour une explication sur \texttt{installeTP.sh}.

\section{Pour s'échauffer}


\begin{enumerate}[label=\textbf{[\alph*]}]
  \setlength\itemsep{1em}

\item \textbf{Concaténer} signifie mettre bout à bout des chaînes de
  caractère. Dans l'exemple, la concaténation des chaînes \texttt{ploum},
  \texttt{tra} et \texttt{lala} doit donner
  \texttt{ploumtralala}. C'est à dire le tableau de caractère
  suivant~:

  \begin{center}
    \begin{tabular}{|c|c|c|c|c|c|c|c|c|c|c|c|c|c}
      \hline
      \texttt{'p'} & \texttt{'l'} & \texttt{'o'} & \texttt{'u'} & \texttt{'m'} & \texttt{'t'} & \texttt{'r'} & \texttt{'a'} & \texttt{'l'} & \texttt{'a'} & \texttt{'l'} & \texttt{'a'} & \texttt{$'\backslash 0'$}  & \\
      \hline
    \end{tabular}
  \end{center}

\item Pour tester les différentes fonctions que l'on va écrire
  (\texttt{mon\_strlen}, \texttt{mon\_strcpy}, \texttt{mon\_strcat},
  \texttt{mon\_strcmp}), pas le choix, il faut pour chacune d'entre
  elles écrire un appel pour chacun des cas possibles. C'est un travail
  long et fastidieux qu'il est pourtant indispensable de faire pour
  s'assurer qu'un programme fonctionne correctement.

\end{enumerate}

\newpage

\section{Implémentation possible pour \texttt{chaines.c}}

\lstinputlisting{listing/chaines_func.c}

\vspace{0.5cm}

Avant de parcourir le \texttt{main} du fichier \texttt{chaines.c},
notons plusieurs choses :

\begin{itemize}
\item \textbf{ATTENTION !} Toutes les fonctions ci-dessus font l'hypothèse que les
  pointeurs qui leur sont passés en paramètre sont \textbf{non nuls}.
\item \textbf{ATTENTION !} Toutes celles qui modifient une chaine de
  caractère considèrent que la mémoire a déjà été allouée. Ex:
  \texttt{mon\_strcat} considère que \texttt{destination} est un
  tableau suffisament grand pour stocker le résultat de la
  concaténation.
\item On remarque que la fonction \texttt{mon\_strcpy} et les
  suivantes n'utilisent pas \texttt{mon\_strlen}. Ce n'est pas par
  masochisme, mais parce que cela permet d'éviter des parcours de
  chaine inutiles ! Pour connaître la taille, il faut regarder chaque
  caractère un par un jusqu'à la fin. C'est une opération coûteuse que
  l'on doit éviter si l'on peut.
\item Si vous lisez cette implémation consciencieusement, vous ne
  manquerez pas de constater que \texttt{mon\_strcmp} est écrite un
  peu légèrement ... En effet, il ne suffit pas qu'un caractère soit
  situé avant un autre dans la table \texttt{ASCII} pour qu'il le
  précède selon l'ordre lexicographique. Par exemple, selon la table
  \texttt{ASCII}, le caractère \texttt{'Z'} est situé avant le
  caractère \texttt{'a'}. Or la lettre z majuscule n'est pas avant la
  lettre a minuscule dans l'alphabet. Pourtant, est-ce une mauvaise
  implémentation ? Elle a le mérite d'être très simple. Par ailleurs,
  elle est peut être suffisante selon le contexte, par exemple si l'on
  sait que l'on ne traîte que des chaînes contenant des lettres
  minuscules. Prenez donc garde à ne pas passer trop de temps à
  perfectionner un programme qui résout déjà 99\% des cas !
\end{itemize}

\vspace{0.5cm}

\lstinputlisting{listing/chaines_main.c}

\vspace{0.5cm}

Résultat de l'exécution de cette implémentation :

\vspace{0.5cm}

\begin{verbatim}
corentin@gazelle:~ $ ./chaine
un mot (pas plus grand que 50 caractères !) à mesurer ?
maison
longueur du mot ('maison') : 6

mon_strcpy(resultat, "copie simple")          resultat = <copie simple>
mon_strcpy(resultat, "")                      resultat = <>
mon_strcat(resultat, "ajout1")                resultat = <ajout1>
mon_strcat(resultat, " ajout2")               resultat = <ajout1 ajout2>
mon_strcat(resultat, "")                      resultat = <ajout1 ajout2>

mon_strcmp("", "")            = 0
mon_strcmp("abc", "abc")      = 0
mon_strcmp("abc", "xbc")      = -1
mon_strcmp("abc", "abC")      = 1
mon_strcmp("a", "abc")        = -1
\end{verbatim}

\section{Fabriquer des phrases}

\begin{enumerate}[label=\textbf{[\alph*]},resume]
  \setlength\itemsep{1em}

\item On a bien sûr pensé, lors de l'écriture de \texttt{phrases.c}, à
  ajouter des espaces entre les chaines. Cela donne donc :

  \begin{lstlisting}
	char Sujet[50]="la petite souris";
	char Verbe[50]="mange";
	char Compl[50]="le gros chat";
	char Phrase[150];

	mon_strcat(Phrase, Sujet);
	mon_strcat(Phrase, " ");
	mon_strcat(Phrase, Verbe);
	mon_strcat(Phrase, " ");
	mon_strcat(Phrase, Compl);
  \end{lstlisting}

\item Pour compter les mots dans la phrase, si on l'a construite
  correctement, il suffit de compter les espaces. Sans oublier de
  rajouter 1 à la fin, puisqu'il y a une espace entre chaque mot, mais
  pas à la fin de la phrase. Implémentation complète de
  \texttt{phrases.c} :

  \vspace{0.5cm}

  \lstinputlisting{listing/phrases.c}

\end{enumerate}

\section{Sacs de billes}


\begin{enumerate}[label=\textbf{[\alph*]},resume]
  \setlength\itemsep{1em}

\item Dans la fonction \texttt{lire\_joueur}, on veut mettre à jour la
  variable contenant les informations d'un joueur avec ce que
  l'utilisateur aura tapé au clavier. On doit donc connaître l'adresse
  où est effectivement stockée cette variable. Or, dans le prototype
  de fonction : \texttt{lire\_joueur(joueur j)}, le paramètre
  \texttt{j} n'est pas un pointeur. C'est donc un passage par
  \textbf{copie}. Si l'on modifie \texttt{j} dans cette fonction,
  c'est la copie fabriquée pour la fonction qui sera modifiée et non
  pas la variable originale.

\item Une implémentation possible de billes :

\vspace{0.5cm}

  \lstinputlisting{listing/billes.c}

\end{enumerate}

\newpage

\section{La command \texttt{diff}}

On considère les fichiers \texttt{d1.c}, \texttt{d2.c} et \texttt{d3.c}
 contenant respectivement :

  \lstinputlisting{listing/d1.c}
  \lstinputlisting{listing/d2.c}
  \lstinputlisting{listing/d3.c}

\vspace{0.5cm}

À première vue, ils semblent identiques. Voici le résultat de
plusieurs commandes \texttt{diff}.

\begin{Verbatim}[xleftmargin=2em]
corentin@gazelle:listing $ diff d1.c d2.c
corentin@gazelle:listing $
\end{Verbatim}

On voit que le diff de \texttt{d1.c} et \texttt{d2.c} se termine
sans rien imprimer sur l'écran. C'est donc que les deux fichiers sont
identiques.

\begin{Verbatim}[xleftmargin=2em]
corentin@gazelle:listing $ diff d1.c d3.c
4c4
< 	printf("Hello, world!");
---
> 	printf("Hello, world!\n");
corentin@gazelle:listing $
\end{Verbatim}

En revanche, le diff de \texttt{d1.c} et \texttt{d3.c} donne la
liste des différences entre les deux fichiers. Plusieurs
explications~:

\begin{description}[leftmargin=!,labelwidth=\widthof{\bfseries gauche/droite}]
\item[gauche/droite] Ici, on appelle \textit{fichier de gauche}
\texttt{d1.c} car c'est le premier paramètre donné à \texttt{diff}. Il
est donc situé à gauche sur la ligne de commande. Par suite,
\texttt{d3.c} est le fichier de droite.
\item[4c4] \textbf{4} $\rightarrow$ ligne 4 de \texttt{d1.c} \\
  \textbf{c} $\rightarrow$ a changé \\
  \textbf{4} $\rightarrow$ correspond désormais à ligne 4 de \texttt{d3.c}
\item[\texttt{<}] Indique que la ligne qui suit se trouve dans le fichier de
  gauche
\item[\texttt{>}] Indique que la ligne qui suit se trouve dans le gichier de
  droite
\item[\texttt{------}] Sépare les lignes situés dans des fichiers différents
\end{description}

\end{document}
