\documentclass[10pt]{article}

\usepackage{graphics}
\usepackage{dirtree}
\usepackage{paracol}
\usepackage{epigraph}
\usepackage{enumitem}
\usepackage{xcolor}
\usepackage{fancyvrb}
\usepackage{calc}

\setlength{\textwidth}{17cm}
\setlength{\oddsidemargin}{-1cm}
\setlength{\evensidemargin}{-1cm}
\setlength{\textheight}{26cm}
\setlength{\parindent}{0pt}
\setlength{\parskip}{0.3ex}

\usepackage{fancyhdr}
\usepackage{epsfig}
\usepackage[utf8]{inputenc}
\usepackage[french]{babel}
\usepackage[T1]{fontenc}
%\usepackage{verbatim}
\usepackage{graphics}
\usepackage{amsmath,amsfonts,amssymb}
\usepackage{listings}
\usepackage{thmbox}
\usepackage{comment}

\oddsidemargin=0cm
\evensidemargin=0cm
\textwidth=17cm
\textheight=25cm
\topmargin=0mm
\hoffset=-6mm
\voffset=-25mm
%\headsep=30pt

\newcounter{numpartie}
\newcounter{exo}
\newcommand{\partie}[2]
{
        \refstepcounter{numpartie}
        \setcounter{exo}{0}
        \vspace{3mm}

        %\noindent{\large\bf Partie \Alph{numpartie}~-~#1}\hfill
        \noindent{\large\bf Partie \Alph{numpartie}~ ~#1}\hfill
        %[\;{\it bar\^eme indicatif : #2}\;]\\
        \\

}

\newenvironment{exercice}{\refstepcounter{exo} \vspace*{1em}\begin{thmbox}[S]{{\bf Exercice \arabic{exo}~:}}}{\end{thmbox}}

\newenvironment{solution}{\vspace*{1em}\begin{thmbox}[S]{{\bf Solution~:}}}{\end{thmbox}}

\newcommand{\IGNORE}[1]{}

\def\EnteteUE{
\noindent{\bf Université Grenoble Alpes} \hfill {\bf DLST}

\vspace{-8pt}

\noindent\hrulefill

\vspace{-2pt}

\noindent{\bf UE INF203} \hfill {\bf Année 2017-18}

\vspace*{.5cm}
}

\newcommand{\PiedDePageUE}[1]{
\fancypagestyle{monstyle}
{
\rhead{}
\lfoot{{\bf\small INF203 - 2016/2017}}
\cfoot{\thepage}
\rfoot{{\it #1}}
\renewcommand{\footrulewidth}{0.5pt}
\renewcommand{\headrulewidth}{0.0pt}
}
\pagestyle{monstyle}
}


\newcounter{question}
\newenvironment{question}[1][]{\refstepcounter{question}
   \textbf{[\bf \alph{question}] {\begin{emph}{#1}\end{emph}}}}{~$\blacksquare$~\\}

\newcommand{\unix}[1]{\hspace*{2cm}{\bf \tt #1}}
\newcommand{\fich}[1]{{\bf \em #1}}
\newcommand{\ecran}[1]{{\tt #1}}
\newcommand{\C}[1]{{\tt #1}}
\newcommand{\cmd}[1]{{\bf #1}}
\newcommand{\home}{\~{}}

\newenvironment{exocomp}{\vspace*{1em}\begin{thmbox}[M]{Exercice complémentaire :~}}{\end{thmbox}}


\begin{document}
\thispagestyle{empty}

\definecolor{mygreen}{rgb}{0,0.6,0}
\definecolor{mygray}{rgb}{0.5,0.5,0.5}
\definecolor{mymauve}{rgb}{0.58,0,0.82}

\lstdefinestyle{customc}{
  belowcaptionskip=1\baselineskip,
  breaklines=true,
  frame=L,
  xleftmargin=\parindent,
  language=C,
  showstringspaces=false,
  basicstyle=\footnotesize\ttfamily,
  keywordstyle=\bfseries\color{green!40!black},
  commentstyle=\itshape\color{purple!40!black},
  identifierstyle=\color{blue},
  stringstyle=\color{orange},
}

\lstdefinestyle{none}{
  belowcaptionskip=1\baselineskip,
  breaklines=true,
  frame=L,
  xleftmargin=\parindent,
  language=bash,
  showstringspaces=false,
  basicstyle=\footnotesize\ttfamily,
  keywordstyle=\bfseries\color{black},
  commentstyle=\itshape\color{black},
  identifierstyle=\color{black},
  stringstyle=\color{black},
}

\lstset{tabsize=3, style=customc,literate=
  {á}{{\'a}}1 {é}{{\'e}}1 {í}{{\'i}}1 {ó}{{\'o}}1 {ú}{{\'u}}1
  {Á}{{\'A}}1 {É}{{\'E}}1 {Í}{{\'I}}1 {Ó}{{\'O}}1 {Ú}{{\'U}}1
  {à}{{\`a}}1 {è}{{\`e}}1 {ì}{{\`i}}1 {ò}{{\`o}}1 {ù}{{\`u}}1
  {À}{{\`A}}1 {È}{{\'E}}1 {Ì}{{\`I}}1 {Ò}{{\`O}}1 {Ù}{{\`U}}1
  {ä}{{\"a}}1 {ë}{{\"e}}1 {ï}{{\"i}}1 {ö}{{\"o}}1 {ü}{{\"u}}1
  {Ä}{{\"A}}1 {Ë}{{\"E}}1 {Ï}{{\"I}}1 {Ö}{{\"O}}1 {Ü}{{\"U}}1
  {â}{{\^a}}1 {ê}{{\^e}}1 {î}{{\^i}}1 {ô}{{\^o}}1 {û}{{\^u}}1
  {Â}{{\^A}}1 {Ê}{{\^E}}1 {Î}{{\^I}}1 {Ô}{{\^O}}1 {Û}{{\^U}}1
  {œ}{{\oe}}1 {Œ}{{\OE}}1 {æ}{{\ae}}1 {Æ}{{\AE}}1 {ß}{{\ss}}1
  {ű}{{\H{u}}}1 {Ű}{{\H{U}}}1 {ő}{{\H{o}}}1 {Ő}{{\H{O}}}1
  {ç}{{\c c}}1 {Ç}{{\c C}}1 {ø}{{\o}}1 {å}{{\r a}}1 {Å}{{\r A}}1
  {€}{{\euro}}1 {£}{{\pounds}}1 {«}{{\guillemotleft}}1
  {»}{{\guillemotright}}1 {ñ}{{\~n}}1 {Ñ}{{\~N}}1 {¿}{{?`}}1
}

\EnteteUE

\begin{center}
  {\large {\bf Corrigé TP3}}
\end{center}

\vspace*{0.5cm}

\section{Arguments de la ligne de commande en C}

\begin{enumerate}[label=\textbf{[\alph*]}]
  \setlength\itemsep{1em}

\item \texttt{argc} représente le nombre d'argument passés au
  programme par la ligne de commande. Par exemple, quand on exécute :

\begin{verbatim}
     ./programme abc toto 123
\end{verbatim}

  Le nombre d'arguments de cette commande, et donc \texttt{argc}, est 4.

  \begin{center}
    \begin{tabular}{|c|}
      \hline
      \texttt{"./programme"} \\
      \hline
      \texttt{"abc"} \\
      \hline
      \texttt{"toto"} \\
      \hline
      \texttt{"123"} \\
      \hline
    \end{tabular}
  \end{center}

\textbf{NB:} On voit que le nom du programme fait partie des
arguments. Il est toujours présent et c'est toujours le premier, à
l'indice 0 du tableau des arguments.

\item Sans plus de modification, le programme affiche seulement le
  signe \texttt{'='} et la valeur à laquelle la somme a été
  initialisée.

\item Une implémentation possible du programme \texttt{somme} :

  \vspace{0.5cm}
  \lstinputlisting{listing/somme.c}

  \newpage
  \section{Lecture depuis un fichier, écriture dans un fichier}

\item Si le nombre d'argument passé est incorrect, la plupart du
  temps, cela signifie que l'utilisateur ne sait pas comment
  fonctionne le programme. On affiche donc un récapitulatif des
  options disponibles. On appelle cela un \textbf{USAGE}.

\begin{verbatim}
    USAGE: ./prog FICHIER
\end{verbatim}

  Implémentation possible pour \texttt{mes\_entrees\_sorties} :

  \vspace{0.5cm}
  \lstinputlisting{listing/mes_entrees_sorties.c}

  \newpage
\item Implémentation possible pour \texttt{mes\_entrees\_sorties} avec
  la consigne complémentaire :

  \vspace{0.5cm}
  \lstinputlisting{listing/mes_entrees_sorties_2.c}

  \newpage
\item On observe que \texttt{cat *.c} affiche successivement tous les
  fichiers C expansés par l'expression \texttt{*.c}. Une modification
  possible de \texttt{mon\_cat.c} pour reproduire ce comportement :

  \vspace{0.5cm}
  \lstinputlisting{listing/mon_cat.c}

\item On observe que \texttt{cat} appelé sans argument lit les données
  à afficher sur l'entrée standard. C'est à dire que lorsqu'on lance
  la commande :

\begin{verbatim}
    cat
\end{verbatim}

  Le terminal se bloque en attente de l'entrée utilisateur. On peut
  alors taper du texte au clavier. Puis, lorsque l'on tape
  \texttt{entrée}, la ligne que l'on vient de terminer est ré-affichée
  par \texttt{cat}. Ce comportement se poursuit jusqu'à que l'on
  interrompe le flux en tapant Ctrl + D. \\

  Reproduire ce comportement est un peu difficile car il n'existe pas
  de moyen simple pour faire un scanf sur une ligne en C, c'est à dire
  jusqu'à un retour chariot. On peut utiliser \texttt{"\%c"} pour un
  caractère, \texttt{"\%s"} pour un mot, mais il n'y a pas de marqueur
  pour la ligne entière. De plus, il est impossible de prévoir à
  l'avance la taille des lignes que l'utilisateur va entrer au
  clavier. Pour se prémunir d'un débordement de tableau, il faut donc
  être très vigilent. Voir à la page suivante une implémentation
  possible de \texttt{cat} qui reproduit le comportement de la
  commande originale~:

  \newpage
  \lstinputlisting{listing/mon_cat_2.c}

  \newpage
  \section{Caractères imprimables}

\item Pour trouver l'ensemble des caractères imprimables, on utilise
  la commande \texttt{man ascii}, qui contient un rappel de la table
  ASCII. On constate que les caractères imprimables forment un bloc
  contigu. Ce sont les caractères compris entre le \texttt{'!'} et le
  \texttt{'~'}. C'est à dire ceux dont le code est compris entre 33 et
  126.

  Voilà une implémentation possible de \texttt{occurences} :

  \vspace{0.5cm}
  \lstinputlisting{listing/occurences.c}


  \newpage
  \section{La commande de la semaine : find}

\item Réponse lorsque j'aurai accès à Turing !

\end{enumerate}
\end{document}
