\documentclass[10pt]{article}

\usepackage{graphics}
\usepackage{dirtree}
\usepackage{paracol}
\usepackage{epigraph}
\usepackage{enumitem}
\usepackage{xcolor}
\usepackage{fancyvrb}
\usepackage{calc}

\setlength{\textwidth}{17cm}
\setlength{\oddsidemargin}{-1cm}
\setlength{\evensidemargin}{-1cm}
\setlength{\textheight}{26cm}
\setlength{\parindent}{0pt}
\setlength{\parskip}{0.3ex}

\usepackage{fancyhdr}
\usepackage{epsfig}
\usepackage[utf8]{inputenc}
\usepackage[french]{babel}
\usepackage[T1]{fontenc}
%\usepackage{verbatim}
\usepackage{graphics}
\usepackage{amsmath,amsfonts,amssymb}
\usepackage{listings}
\usepackage{thmbox}
\usepackage{comment}

\oddsidemargin=0cm
\evensidemargin=0cm
\textwidth=17cm
\textheight=25cm
\topmargin=0mm
\hoffset=-6mm
\voffset=-25mm
%\headsep=30pt

\newcounter{numpartie}
\newcounter{exo}
\newcommand{\partie}[2]
{
        \refstepcounter{numpartie}
        \setcounter{exo}{0}
        \vspace{3mm}

        %\noindent{\large\bf Partie \Alph{numpartie}~-~#1}\hfill
        \noindent{\large\bf Partie \Alph{numpartie}~ ~#1}\hfill
        %[\;{\it bar\^eme indicatif : #2}\;]\\
        \\

}

\newenvironment{exercice}{\refstepcounter{exo} \vspace*{1em}\begin{thmbox}[S]{{\bf Exercice \arabic{exo}~:}}}{\end{thmbox}}

\newenvironment{solution}{\vspace*{1em}\begin{thmbox}[S]{{\bf Solution~:}}}{\end{thmbox}}

\newcommand{\IGNORE}[1]{}

\def\EnteteUE{
\noindent{\bf Université Grenoble Alpes} \hfill {\bf DLST}

\vspace{-8pt}

\noindent\hrulefill

\vspace{-2pt}

\noindent{\bf UE INF203} \hfill {\bf Année 2016-17}

\vspace*{.5cm}
}

\newcommand{\PiedDePageUE}[1]{
\fancypagestyle{monstyle}
{
\rhead{}
\lfoot{{\bf\small INF203 - 2016/2017}}
\cfoot{\thepage}
\rfoot{{\it #1}}
\renewcommand{\footrulewidth}{0.5pt}
\renewcommand{\headrulewidth}{0.0pt}
}
\pagestyle{monstyle}
}


\newcounter{question}
\newenvironment{question}[1][]{\refstepcounter{question}
   \textbf{[\bf \alph{question}] {\begin{emph}{#1}\end{emph}}}}{~$\blacksquare$~\\}

\newcommand{\unix}[1]{\hspace*{2cm}{\bf \tt #1}}
\newcommand{\fich}[1]{{\bf \em #1}}
\newcommand{\ecran}[1]{{\tt #1}}
\newcommand{\C}[1]{{\tt #1}}
\newcommand{\cmd}[1]{{\bf #1}}
\newcommand{\home}{\~{}}

\newenvironment{exocomp}{\vspace*{1em}\begin{thmbox}[M]{Exercice complémentaire :~}}{\end{thmbox}}


\begin{document}
\thispagestyle{empty}

\makeatletter
\lst@InstallKeywords k{attributes}{attributestyle}\slshape{attributestyle}{}ld
\makeatother

\definecolor{mygreen}{rgb}{0,0.3,0}
\definecolor{mygray}{rgb}{0.5,0.5,0.5}
\definecolor{mymauve}{rgb}{0.9,0,0.4}

\lstdefinestyle{customc}{
  belowcaptionskip=1\baselineskip,
  breaklines=true,
  frame=L,
  xleftmargin=\parindent,
  language=C,
  showstringspaces=false,
  basicstyle=\footnotesize\ttfamily,
  keywordstyle=\bfseries\color{green!40!black},
  commentstyle=\itshape\color{purple!40!black},
  identifierstyle=\color{blue},
  stringstyle=\color{orange},
}

\lstdefinestyle{transitionc}{
  belowcaptionskip=1\baselineskip,
  breaklines=true,
  frame=L,
  xleftmargin=\parindent,
  language=C,
  showstringspaces=false,
  basicstyle=\footnotesize\ttfamily,
  keywordstyle=\bfseries\color{green!40!black},
  commentstyle=\itshape\color{purple!40!black},
  identifierstyle=\color{blue},
  stringstyle=\color{orange},
  moreattributes={Q1, Q2, Q3, Q4},
  attributestyle = \bfseries\color{mymauve},
}

\lstdefinestyle{none}{
  belowcaptionskip=1\baselineskip,
  breaklines=true,
  frame=L,
  xleftmargin=\parindent,
  language=bash,
  showstringspaces=false,
  basicstyle=\footnotesize\ttfamily,
  keywordstyle=\bfseries\color{black},
  commentstyle=\itshape\color{black},
  identifierstyle=\color{black},
  stringstyle=\color{black},
}

\lstset{tabsize=3, style=customc,literate=
  {á}{{\'a}}1 {é}{{\'e}}1 {í}{{\'i}}1 {ó}{{\'o}}1 {ú}{{\'u}}1
  {Á}{{\'A}}1 {É}{{\'E}}1 {Í}{{\'I}}1 {Ó}{{\'O}}1 {Ú}{{\'U}}1
  {à}{{\`a}}1 {è}{{\`e}}1 {ì}{{\`i}}1 {ò}{{\`o}}1 {ù}{{\`u}}1
  {À}{{\`A}}1 {È}{{\'E}}1 {Ì}{{\`I}}1 {Ò}{{\`O}}1 {Ù}{{\`U}}1
  {ä}{{\"a}}1 {ë}{{\"e}}1 {ï}{{\"i}}1 {ö}{{\"o}}1 {ü}{{\"u}}1
  {Ä}{{\"A}}1 {Ë}{{\"E}}1 {Ï}{{\"I}}1 {Ö}{{\"O}}1 {Ü}{{\"U}}1
  {â}{{\^a}}1 {ê}{{\^e}}1 {î}{{\^i}}1 {ô}{{\^o}}1 {û}{{\^u}}1
  {Â}{{\^A}}1 {Ê}{{\^E}}1 {Î}{{\^I}}1 {Ô}{{\^O}}1 {Û}{{\^U}}1
  {œ}{{\oe}}1 {Œ}{{\OE}}1 {æ}{{\ae}}1 {Æ}{{\AE}}1 {ß}{{\ss}}1
  {ű}{{\H{u}}}1 {Ű}{{\H{U}}}1 {ő}{{\H{o}}}1 {Ő}{{\H{O}}}1
  {ç}{{\c c}}1 {Ç}{{\c C}}1 {ø}{{\o}}1 {å}{{\r a}}1 {Å}{{\r A}}1
  {€}{{\euro}}1 {£}{{\pounds}}1 {«}{{\guillemotleft}}1
  {»}{{\guillemotright}}1 {ñ}{{\~n}}1 {Ñ}{{\~N}}1 {¿}{{?`}}1
}

\EnteteUE

\begin{center}
  {\large {\bf Corrigé TP12}}
\end{center}

\vspace*{0.5cm}

\begin{center}
  {\large {\bf Développement d'un interpréteur de commandes (2/2)}}
\end{center}

\vspace*{0.5cm}

\begin{enumerate}[label=\textbf{[\alph*]}]
  \setlength\itemsep{1em}

\item Ce n'est vraiment pas la peine de faire plus de scripts de test
  que ce qui est déjà fourni.

\item On est en train de chercher où est appelé la fonction
  \texttt{decode\_entree}. On pense immédiatement à la commande
  \texttt{grep} que nous avons vue au début du semestre. On se place
  donc à la racine de notre projet et on tape :

\begin{verbatim}
    $ grep -Rn decode_entree
    listing/commandes.c:81:int decode_entree(char *mot) {
    listing/commandes.c:162:    entree = decode_entree(bout_de_ligne);
\end{verbatim}

  Grep nous renvoie donc, en couleur s'il vous plait, les deux endroit
  où il a trouver le nom de la fonction. Ligne 81 du fichier
  commandes.c, c'est la déclaration et ligne 162, c'est le seul appel
  pour l'instant dans le projet.

  \vspace{0.2cm}
  Voilà une implémentation possible de cette fonction :

  \lstinputlisting{listing/extrait1.c}

\item La difficulté de l'implémentation des fonctions manquantes de
  \texttt{commandes.c} est que vous n'êtes guidés qu'avec partimonie
  sur les endroits à compléter. Il faut donc bien comprendre
  l'objectif du programme pour savoir où l'on va.

  \vspace{0.2cm}
  Trois choses à compléter, donc. La fonction \texttt{decode\_entree}
  a été vue à la question précédente.

  \vspace{0.2cm}
  La fonction
  \texttt{init\_automate\_commandes} consiste à remplir la table de
  transition de l'automate pour le faire correspondre à l'automate de
  gestion du if vu en TD. Ensuite, il faut également remplir la table
  de sortie pour savoir dans quels cas la ligne courante doit être
  exécutée. Enfin, il faut paramétrer l'état courant de l'automate à
  l'état initial. Cette façon de faire est un peu différente de celle
  dont on a l'habitude : l'état courant est à présent stocké
  directement dans la structure de l'automate.


  \vspace{0.2cm}
  Enfin, il faut compléter une partie de la fonction
  \texttt{analyse\_commande\_interne}. C'est à cet endroit que l'on va
  insérer l'algorithme usuel pour faire marcher l'automate, c'est à
  dire :
  \begin{itemize}
  \item Récupérer l'entrée
  \item Calculer l'état suivant en fonction de l'état courant et de
    l'entrée
  \item Calculer la sortie de la même manière
  \item Mettre à jour l'état courant avec l'état suivant
  \end{itemize}

  \vspace{0.2cm}
  Voilà une implémentation possible des deux dernières fonctions.

  \lstinputlisting{listing/extrait2.c}

\item On note que seules deux fonctions de \texttt{commandes.c} ont
  leur prototype déclaré dans \texttt{commandes.h}. C'est parce que
  les autres fonctions ne sont utilisées que dans
  \texttt{commandes.c}. Pas la peine de les exposer au monde
  entier. On a donc deux fonctions dont ont cherche les endroits où
  elles sont appelées dans le code. Pour ça, comme pour la première
  question, rien de tel qu'un bon petit grep !

  \vspace{0.2cm}
  \begin{itemize}
  \item \texttt{analyse\_commande\_interne} est appelée dans
    \texttt{interpreteur.c}, ligne 54.
  \item \texttt{init\_automate\_commandes} est appelée dans
    \texttt{interpreteur.c}, ligne 98.
  \end{itemize}
  \newpage

\item Une fois testé notre interpréteur, à présent avec quelques
  scripts contenant des if/then/else, on passe à la suite, c'est à
  dire à l'historique.

  \vspace{0.2cm}
  Comme nous avons expliqué tout le principe de l'historique en TD et
  que le schéma récapitulatif dans le sujet de TP est bien fait, voici
  sans plus attendre une implémentation possible pour
  \texttt{lignes.c}.

  \vspace{0.2cm}
  Dans \texttt{lignes.h}, on rajoute :

  \lstinputlisting{listing/extrait3.c}

  \vspace{0.2cm}
  Dans \texttt{lignes.c}, on rajoute :

  \lstinputlisting{listing/extrait4.c}

  On voit que le mécanisme de l'historique, qui n'est pas si facile à
  comprendre, se plie en fait en quelques fonctions assez courtes.

  \vspace{0.2cm}
  Reste à aborder le problème \textbf{PRINCIPAL} de cette
  question. Les fonctions sont maintenant faites. Fastoche. Et en
  plus, ça compile ! Trop easy ! Mais comment faire pour tester ? En
  effet, nous venons d'implémenter l'historique alors que nous n'avons
  pas encore intégrer la gestion du while dans notre
  interpréteur. Comment donc vérifier que nos fonctions sont justes
  alors qu'elles ne sont pour l'instant appellées nulle part dans le
  programme ? Eh bien, pas de miracle, il faut faire des tests
  unitaires, c'est à dire un main spécial dont le but est uniquement
  d'appeler nos fonctions dans tous les cas possibles.

  \vspace{0.2cm}
  On rajoute au \texttt{Makefile} les lignes suivantes :

  \lstinputlisting{listing/extrait5}

  \vspace{0.2cm}
  Et on crée le fichier \texttt{test\_historique.c} avec le contenu
  suivant :

  \lstinputlisting{listing/test_historique.c}

  On peut maintenant compiler avec la commande \texttt{make test} et
  exécuter \texttt{test} pour vérifier que l'historique marche
  bien. Et tout ça dans la bonne humeur car on n'oublie pas que :

  \begin{center}
    {\large {\bf TESTER N'EST JAMAIS DU TEMPS PERDU}}
  \end{center}
  \newpage

\item Voilà le schéma de l'automate pour le while, qu'il faudra
  incorporer dans l'automate principal de notre interpréteur.

  \begin{center}
    ça arrive !
  \end{center}

\item L'implémentation du while est donc une sorte de courronnement de
  votre année. Commençons par prêter un oeil à tout ce qu'il n'y a PAS
  besoin de faire.

  \begin{itemize}
  \item Les constantes ATTENTE\_DO\_VRAI, ATTENTE\_DO\_FAUX, DANS\_DO,
    ATTENTE\_DONE sont déjà présentes pour représenter les états de
    l'automate du while.
  \item Les constantes WHILE, DO, DONE, WHILE\_VRAI, WHILE\_FAUX sont
    déjà présentes pour représenter les entrées de l'automate. On fait
    attention : comme pour le IF, l'entrée WHILE est temporaire, elle
    doit être "résolue" en WHILE\_VRAI ou WHILE\_FAUX selon la condition
    qui suit.
  \item Le tableau des mnémoniques contient déjà les correspondances
    pour le while. La fonction \texttt{decode\_entree} n'a donc pas
    besoin d'être modifiée.
  \item \texttt{selectionne\_alternative} et
    \texttt{extrait\_premier\_mot} n'ont pas besoin d'être modifiées,
    on n'y touchera pas du TP.

  \end{itemize}

  Voilà le boulot à effectuer :

  \begin{itemize}
  \item Incorporer les transitions de l'automate du while dans notre
    automate qui ne gère pour l'instant que le if. Cela consiste donc
    à modifier la fonction \texttt{init\_automate}.
  \item Modifier \texttt{analyse\_commande\_interne} pour effectuer la
    "résolution" du mnémonique while en WHILE\_VRAI ou
    WHILE\_FAUX. Juste en dessous du if qui appelle
    \texttt{selectionne\_alternative}.
  \item Appeler la fonction \texttt{obtenir\_numero\_ligne} losrque
    l'on détecte un while, qu'il soit vrai ou faux, pour connaître son
    numéro de ligne. Puis, enregistrer ce numéro de ligne dans
    l'automate, au registre \texttt{A->ligne\_boucle}.
  \item Appeler la fonction \texttt{aller\_a\_la\_ligne} lorsque la
    sortie de l'automate est BOUCLE, afin de revenir à la ligne du
    while, qu'on ira chercher dans l'automate, puisqu'on l'a
    enregistrée précédemment.
  \item Ne pas oublier d'initialiser l'historique avec le descripteur
    de fichier correspondant au script dans \texttt{interpreteur.c}, à
    l'aide de la fonction \texttt{init\_historique}.
  \end{itemize}

  En un mot comme en cent, voilà les nouvelles implémentations pour
  \texttt{init\_automate\_commandes}.

  \lstinputlisting{listing/extrait6.c}

  \vspace{0.2cm}
  Et la fonction \texttt{analyse\_commande\_interne}.

  \lstinputlisting{listing/extrait7.c}

\end{enumerate}

\end{document}
