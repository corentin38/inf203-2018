\documentclass[10pt]{article}

\usepackage{graphics}
\usepackage{dirtree}
\usepackage{paracol}
\usepackage{epigraph}
\usepackage{enumitem}
\usepackage{xcolor}
\usepackage{fancyvrb}
\usepackage{calc}

\setlength{\textwidth}{17cm}
\setlength{\oddsidemargin}{-1cm}
\setlength{\evensidemargin}{-1cm}
\setlength{\textheight}{26cm}
\setlength{\parindent}{0pt}
\setlength{\parskip}{0.3ex}

\usepackage{fancyhdr}
\usepackage{epsfig}
\usepackage[utf8]{inputenc}
\usepackage[french]{babel}
\usepackage[T1]{fontenc}
%\usepackage{verbatim}
\usepackage{graphics}
\usepackage{amsmath,amsfonts,amssymb}
\usepackage{listings}
\usepackage{thmbox}
\usepackage{comment}

\oddsidemargin=0cm
\evensidemargin=0cm
\textwidth=17cm
\textheight=25cm
\topmargin=0mm
\hoffset=-6mm
\voffset=-25mm
%\headsep=30pt

\newcounter{numpartie}
\newcounter{exo}
\newcommand{\partie}[2]
{
        \refstepcounter{numpartie}
        \setcounter{exo}{0}
        \vspace{3mm}

        %\noindent{\large\bf Partie \Alph{numpartie}~-~#1}\hfill
        \noindent{\large\bf Partie \Alph{numpartie}~ ~#1}\hfill
        %[\;{\it bar\^eme indicatif : #2}\;]\\
        \\

}

\newenvironment{exercice}{\refstepcounter{exo} \vspace*{1em}\begin{thmbox}[S]{{\bf Exercice \arabic{exo}~:}}}{\end{thmbox}}

\newenvironment{solution}{\vspace*{1em}\begin{thmbox}[S]{{\bf Solution~:}}}{\end{thmbox}}

\newcommand{\IGNORE}[1]{}

\def\EnteteUE{
\noindent{\bf Université Grenoble Alpes} \hfill {\bf DLST}

\vspace{-8pt}

\noindent\hrulefill

\vspace{-2pt}

\noindent{\bf UE INF203} \hfill {\bf Année 2016-17}

\vspace*{.5cm}
}

\newcommand{\PiedDePageUE}[1]{
\fancypagestyle{monstyle}
{
\rhead{}
\lfoot{{\bf\small INF203 - 2016/2017}}
\cfoot{\thepage}
\rfoot{{\it #1}}
\renewcommand{\footrulewidth}{0.5pt}
\renewcommand{\headrulewidth}{0.0pt}
}
\pagestyle{monstyle}
}


\newcounter{question}
\newenvironment{question}[1][]{\refstepcounter{question}
   \textbf{[\bf \alph{question}] {\begin{emph}{#1}\end{emph}}}}{~$\blacksquare$~\\}

\newcommand{\unix}[1]{\hspace*{2cm}{\bf \tt #1}}
\newcommand{\fich}[1]{{\bf \em #1}}
\newcommand{\ecran}[1]{{\tt #1}}
\newcommand{\C}[1]{{\tt #1}}
\newcommand{\cmd}[1]{{\bf #1}}
\newcommand{\home}{\~{}}

\newenvironment{exocomp}{\vspace*{1em}\begin{thmbox}[M]{Exercice complémentaire :~}}{\end{thmbox}}


\begin{document}
\thispagestyle{empty}

\makeatletter
\lst@InstallKeywords k{attributes}{attributestyle}\slshape{attributestyle}{}ld
\makeatother

\definecolor{mygreen}{rgb}{0,0.3,0}
\definecolor{mygray}{rgb}{0.5,0.5,0.5}
\definecolor{mymauve}{rgb}{0.9,0,0.4}

\lstdefinestyle{customc}{
  belowcaptionskip=1\baselineskip,
  breaklines=true,
  frame=L,
  xleftmargin=\parindent,
  language=C,
  showstringspaces=false,
  basicstyle=\footnotesize\ttfamily,
  keywordstyle=\bfseries\color{green!40!black},
  commentstyle=\itshape\color{purple!40!black},
  identifierstyle=\color{blue},
  stringstyle=\color{orange},
}

\lstdefinestyle{transitionc}{
  belowcaptionskip=1\baselineskip,
  breaklines=true,
  frame=L,
  xleftmargin=\parindent,
  language=C,
  showstringspaces=false,
  basicstyle=\footnotesize\ttfamily,
  keywordstyle=\bfseries\color{green!40!black},
  commentstyle=\itshape\color{purple!40!black},
  identifierstyle=\color{blue},
  stringstyle=\color{orange},
  moreattributes={Q1, Q2, Q3, Q4},
  attributestyle = \bfseries\color{mymauve},
}

\lstdefinestyle{none}{
  belowcaptionskip=1\baselineskip,
  breaklines=true,
  frame=L,
  xleftmargin=\parindent,
  language=bash,
  showstringspaces=false,
  basicstyle=\footnotesize\ttfamily,
  keywordstyle=\bfseries\color{black},
  commentstyle=\itshape\color{black},
  identifierstyle=\color{black},
  stringstyle=\color{black},
}

\lstset{tabsize=3, style=customc,literate=
  {á}{{\'a}}1 {é}{{\'e}}1 {í}{{\'i}}1 {ó}{{\'o}}1 {ú}{{\'u}}1
  {Á}{{\'A}}1 {É}{{\'E}}1 {Í}{{\'I}}1 {Ó}{{\'O}}1 {Ú}{{\'U}}1
  {à}{{\`a}}1 {è}{{\`e}}1 {ì}{{\`i}}1 {ò}{{\`o}}1 {ù}{{\`u}}1
  {À}{{\`A}}1 {È}{{\'E}}1 {Ì}{{\`I}}1 {Ò}{{\`O}}1 {Ù}{{\`U}}1
  {ä}{{\"a}}1 {ë}{{\"e}}1 {ï}{{\"i}}1 {ö}{{\"o}}1 {ü}{{\"u}}1
  {Ä}{{\"A}}1 {Ë}{{\"E}}1 {Ï}{{\"I}}1 {Ö}{{\"O}}1 {Ü}{{\"U}}1
  {â}{{\^a}}1 {ê}{{\^e}}1 {î}{{\^i}}1 {ô}{{\^o}}1 {û}{{\^u}}1
  {Â}{{\^A}}1 {Ê}{{\^E}}1 {Î}{{\^I}}1 {Ô}{{\^O}}1 {Û}{{\^U}}1
  {œ}{{\oe}}1 {Œ}{{\OE}}1 {æ}{{\ae}}1 {Æ}{{\AE}}1 {ß}{{\ss}}1
  {ű}{{\H{u}}}1 {Ű}{{\H{U}}}1 {ő}{{\H{o}}}1 {Ő}{{\H{O}}}1
  {ç}{{\c c}}1 {Ç}{{\c C}}1 {ø}{{\o}}1 {å}{{\r a}}1 {Å}{{\r A}}1
  {€}{{\euro}}1 {£}{{\pounds}}1 {«}{{\guillemotleft}}1
  {»}{{\guillemotright}}1 {ñ}{{\~n}}1 {Ñ}{{\~N}}1 {¿}{{?`}}1
}

\EnteteUE

\begin{center}
  {\large {\bf Corrigé TP11}}
\end{center}

\vspace*{0.5cm}

\begin{center}
  {\large {\bf Développement d'un interpréteur de commandes (1/2)}}
\end{center}

\vspace*{0.5cm}

Ce TP est sans doute le plus difficile de l'année et celui qui fera
appel au champ le plus large de vos connaissances. Il est aussi le plus
intéressant et le plus complet :

\begin{itemize}
\item Il nécessite de lire et de comprendre du code déjà écrit
\item Il nécessite d'utiliser des fonctions sans s'attarder sur leur
  implémentation.
\item Il vous demande de vous concentrer durant deux séances sur le
  même projet. Ce n'est donc pas un projet jetable, ``code and
  forget'' comme tous les TPs précédents.
\end{itemize}

Il n'existe pas de méthode absolue pour découvrir et comprendre,
\textit{explorer} une base de code existante. Vous allez ainsi
rencontrer des problèmes de la vie réelle :

\begin{itemize}
\item Trouver où, mais \textbf{OÙ} cette punaise de fonction est déclarée
\item Comprendre les noms obscurs des variables et leur signification
\item Voir que l'indentation aide \textbf{VRAIMENT} à lire du code
\item Voir qu'un Makefile est \textbf{INDISPENSABLE} lorsqu'on aborde un projet
  que l'on ne connait pas
\item etc.
\end{itemize}

Tout cela fait partie du boulot d'un informaticien. N'oubliez pas qu'un
développeur passe plus de temps à lire et explorer du code existant
qu'à en écrire. Et à ce petit jeu, le gagnant et souvent celui qui a
les meilleurs outils (et qui sait s'en servir).

\begin{enumerate}[label=\textbf{[\alph*]}]
  \setlength\itemsep{1em}

\item On jette un oeil aux fichier \texttt{lignes.c} et
  \texttt{variables.c} et on découvre avec gravité que toutes les
  fonctions sont vides. Il va falloir les compléter ...

\item Une implémentation possible, dans le fichier \texttt{lignes.c},
  de la fonction \texttt{lire\_ligne\_fichier} :

  \lstinputlisting{listing/extrait1.c}

\item Après avoir complété \texttt{lire\_ligne\_fichier}, on compile
  en utilisant \texttt{make}, puis on teste notre interpreteur sur le
  fichier de commande \texttt{test\_1\_lecture\_ligne.sh}.
  \vspace{0.2cm}

  On voit que l'interpreteur s'exécute bien sur chacune des lignes,
  mais qu'il n'exécute que la commande de base, sans prendre en compte
  les paramètres.
  \vspace{0.2cm}
  \newpage

\item Voici une implémentation possible pour les fonctions de
  \texttt{variables\_base.h} :

  \lstinputlisting{listing/extrait2.c}

  \textbf{NB:} On note plusieurs choses :
  \begin{itemize}
  \item On initialise juste en plaçant \texttt{ens->nb} à 0. Comme ça,
    le programme saura que ce n'est pas la peine d'aller consulter le
    tableau, puisqu'il n'y a pas de variable enregistrée. Pas la peine
    d'initialiser chaque valeur du tableau à NULL.
  \item Dans \texttt{ajouter\_variable}, on n'oublie pas de vérifier
    qu'on a pas atteint le nombre maximum de variables possible avant
    d'en insérer une nouvelle.
  \item Dans \texttt{nom\_variable}, on vérifie quand même que
    l'indice donné est cohérent, c'est à dire positif ou nul, et
    inférieur au nombre maximum de variables possibles.
  \item Dans \texttt{valeur\_variable}, voyez comme on initialise la
    valeur à retourner : \texttt{char *valeur = const\_empty;}

    On a déclaré la chaîne vide ("") comme une variable globale, donc
    à l'extérieur de toute fonction pour être sûr que cette allocation
    de mémoire (1 octet, avec '$\backslash$0' dedans) ne soit pas détruite en
    sortant de la fonction.
  \end{itemize}

\item On attaque maintenant la fonction
  \texttt{trouver\_et\_appliquer\_affectation\_variable}. L'algorithme à
  suivre pour implémenter cette fonction est bien décrit par une
  superbe animation dans le TD que nous avons fait. Pour rappel,
  l'objectif est de chercher dans la ligne une déclaration de
  variables, comme : \texttt{mavariable=42} avec potentiellement des
  espaces avant, mais pas après. Ensuite, une fois l'affectation
  reconnue, il faut enregistrer la variable dans la table en appelant
  les fonctions que nous avons déjà codées. Voilà tout de suite une
  implémentation possible, à la mode des automates avec plein de switch.

  \lstinputlisting{listing/extrait3.c}

  \vspace{0.2cm}
  \textbf{NB:} ici aussi, plusieurs choses à noter :
  \begin{itemize}
  \item L'utilisation d'une enum pour noter tous les états possibles
    au lieu des \texttt{\#define} que vous connaissez. Une enum est un
    peu comme une déclaration automatique de constantes
    numérotées.

    \begin{lstlisting}
      enum { ESP_DEB, NOM, EGAL, VAL, ERR } etat;
    \end{lstlisting}

    \texttt{etat} est une variable de type enum, qui peut
    prendre les valeurs \texttt{ESP\_DEB}, \texttt{NOM},
    \texttt{EGAL}, \texttt{VAL} ou \texttt{ERR}. En fait, en
    pratique, ces constantes sont des entiers, à partir de 0. Donc
    \texttt{ESP\_DEB} = 0, \texttt{NOM} = 1, etc.. On peut donc
    utiliser la variable \texttt{etat} dans un switch.
  \item On a découvert une affectation de variable valide si on
    arrive au bout de la ligne est qu'on est dans l'état
    \texttt{VAL}.
  \item On arrête le parcours dans le cas où on détecte une syntaxe
    invalide d'affectation, c'est à dire lorsqu'on entre dans l'état
    \texttt{ERR}.
  \item Notez comme les caractères (qui ne sont que des nombres
    représentant un caractère selon la table ASCII) se prêtent bien
    aux switch ...
  \end{itemize}

  \textbf{NB2: } Il faut également voir les limitations de
  l'implémentation proposée.
  \begin{itemize}
  \item Elle ne prend pas en compte les guillemets (ce n'était pas
    dans la consigne, mais il n'est pas interdit de faire plus que ce
    qui est demandé !)
  \item Elle ne prend en compte que les espaces comme séparateurs. Les
    tabulations (caractère \texttt{$\backslash$n}) la font donc
    planter
  \item Elle est assez longue (60 lignes). C'est peut être le symptôme
    que l'on essaye de faire trop de choses en même temps.
  \end{itemize}

\item Voilà la seconde fonction compliquée du TP, vue également en
  TD. \texttt{appliquer\_expansion\_variables} doit remplacer toutes
  les utilisations de variable dans la chaîne (\texttt{\$mavariable}
  séparé avant et après par des espaces) par leur valeur réelle ou une
  chaîne vide si la variable appelée n'existe pas. Voilà une
  implémentation toujours à la mode des automates, avec des switch.

  \lstinputlisting{listing/extrait4.c}

  \vspace{0.2cm}
  \textbf{NB: } Encore des notes pour cette fonction.
  \begin{itemize}
  \item \texttt{i\_o}, \texttt{i\_e} et \texttt{i\_n} sont
    respectivement les indices du caractère courant pour les chaînes
    \texttt{ligne\_originale}, \texttt{ligne\_expansee} et
    \texttt{nom}.
  \item Cette fois ci, on utilise la fonction \texttt{isalnum} qu'on
    trouve dans le header \texttt{ctype.h} pour déterminer si le
    caractère courant est alpha numérique, ou autre chose.
  \item On note aussi que l'on peut continuer le parcours de la
    chaîne de caractère un cran après le caractère '$\backslash$0'
    dans le cas où l'on est encore dans l'état expansion lorsqu'on
    l'aborde. Ce hack horrible a le mérite de nous faire remarquer
    qu'un break ne termine que l'expression immédiatement englobante,
    dans ce cas le switch. Le while qui englobe le switch devient par
    conséquent impossible à interrompre sans sortir préalablement du
    switch.
  \end{itemize}

  \vspace{0.2cm}
  Comme de nombreux étudiants, plutôt que de s'adonner aux plaisirs de
  l'imbrication de switch, ont sûrement choisi d'implémenter cette
  fonction avec des if en cascade, voilà une implémentation à cette
  sauce. On remarquera que pour ne pas s'y perdre, on a créer une
  fonction intermédiaire \texttt{expand}.

  \lstinputlisting{listing/extrait5.c}

  \vspace{0.2cm}
  Une dernière chose ... Les fonctions de gestion des variables sont
  véritablement complexes et difficiles à coder. Comme nous ne sommes
  pas des génies, nous ne pouvons pas y arriver du premier coup. Il
  est presque obligatoire de procéder par essais/erreurs. Pour cela,
  le plus facile est de se faire des tests unitaires afin de se
  concentrer sur LA fonction que l'on est en train de coder et de
  s'assurer qu'elle rempli bien son rôle dans tous les cas possibles.

  \vspace{0.2cm}
  Voilà un main permettant de tester les fonctions de
  \texttt{variables.c}.

  \lstinputlisting{listing/src/main_unit_test.c}

  \vspace{0.2cm}
  Ces tests dépassent le cadre de ce TP, mais vous pouvez observer,
  dans le \texttt{main} combien il est facile de rajouter des cas de
  test pour vérifier que nos fonctions produisent bien le résultat
  escompté même dans les situations les plus tordues ... Ce programme
  produit la sortie suivante lorsque les fonctions de
  \texttt{variables.c} sont implémentées correctement :

\begin{verbatim}
corentin@gazelle:src $ ./main_unit_test
TOTO=5
TITI=12
TATA=Chantons gaiement !


[ OK ] test case <TOTO=5> OK
[ OK ] test case <     TOTO=5> OK

[ OK ] test case <     TOTO =5> OK
[ OK ] test case <     TOTO= 5> OK
[ OK ] test case <     TOTO=5 > OK

[ OK ] test case <$TOTO> OK
[ OK ] test case <aeofubaoeu $TOTO aeufbaeub> OK
[ OK ] test case <aeofubaoeu $TOTO> OK
[ OK ] test case <aeofubaoeu $ aefae> OK

Testing with : argc=5 argv=[/bin/monprog, argument1, argument2, argument3, argument4]
[ OK ] $# = <4>
[ OK ] $* = </bin/monprog argument1 argument2 argument3 argument4>
[ OK ] $0 = </bin/monprog>
[ OK ] $1 = <argument1>
[ OK ] $2 = <argument2>
[ OK ] $3 = <argument3>
[ OK ] $4 = <argument4>
$
\end{verbatim}

\item Après les fonctions sur les variables, la fonction
  \texttt{decouper\_ligne} paraît être une promenade de santée. Elle a
  été également largement décrite en TD, place à une implémentation
  possible.

  \lstinputlisting{listing/extrait6.c}

  \vspace{0.2cm}

\item Nous passons aux variables automatiques. Toute cette partie du
  TP est ``complémentaire'', c'est à dire facultative. Elle donne un
  exemple de l'utilisation de la fonction \texttt{strcat}.

  \vspace{0.2cm}
  On rappelle que toutes les variables en Shell sont des chaînes de
  caractères. Même les nombres sont donc stockés sous cette forme et
  ne sont interprétés comme nombre que lorsqu'on les donne à des
  fonctions comme \texttt{test}, qui s'occupe alors de faire la
  conversion.

  \vspace{0.2cm}
  \texttt{\$\#} représente le nombre de paramètres passé à la
  commande. Dans notre cas, cela correspondra donc à
  \texttt{argc - 1}, vu que le Shell ne compte pas le nom du programme
  comme un paramètre.

  \vspace{0.2cm}
  Voilà donc une implémentation de la fonction
  \texttt{affecter\_variables\_automatiques}.

  \lstinputlisting{listing/extrait7.c}



%
%  \vspace{0.2cm}
%
%  \lstinputlisting{listing/Digicode_V1/automate.c}
%
%
%  \begin{center}
%    \begin{tabular}{|l|c|c|c|c|c|}
%      \hline
%      & 1 & 2 & 3 & 4 & 0, 5, 6, 7, 8, 9 \\
%      \hline
%      \textbf{Q0} & Q1 & Q0 & Q0 & Q0 & Q0 \\
%      \textbf{Q1} & Q1 & Q2 & Q0 & Q0 & Q0 \\
%      \textbf{Q2} & Q1 & Q0 & Q3 & Q0 & Q0 \\
%      \textbf{Q3} & Q1 & Q0 & Q0 & Q4 & Q0 \\
%      \textbf{Q4} & Q1 & Q0 & Q0 & Q0 & Q0 \\
%      \hline
%    \end{tabular}
%  \end{center}
%
%  \lstset{style=transitionc}
%  \lstinputlisting{listing/Digicode_V3/automate.c}
%

\end{enumerate}

\end{document}
