\documentclass[10pt]{article}

\usepackage{graphics}
\usepackage{dirtree}
\usepackage{paracol}
\usepackage{epigraph}
\usepackage{enumitem}
\usepackage{xcolor}
\usepackage{fancyvrb}
\usepackage{calc}

\setlength{\textwidth}{17cm}
\setlength{\oddsidemargin}{-1cm}
\setlength{\evensidemargin}{-1cm}
\setlength{\textheight}{26cm}
\setlength{\parindent}{0pt}
\setlength{\parskip}{0.3ex}

\usepackage{fancyhdr}
\usepackage{epsfig}
\usepackage[utf8]{inputenc}
\usepackage[french]{babel}
\usepackage[T1]{fontenc}
%\usepackage{verbatim}
\usepackage{graphics}
\usepackage{amsmath,amsfonts,amssymb}
\usepackage{listings}
\usepackage{thmbox}
\usepackage{comment}

\oddsidemargin=0cm
\evensidemargin=0cm
\textwidth=17cm
\textheight=25cm
\topmargin=0mm
\hoffset=-6mm
\voffset=-25mm
%\headsep=30pt

\newcounter{numpartie}
\newcounter{exo}
\newcommand{\partie}[2]
{
        \refstepcounter{numpartie}
        \setcounter{exo}{0}
        \vspace{3mm}

        %\noindent{\large\bf Partie \Alph{numpartie}~-~#1}\hfill
        \noindent{\large\bf Partie \Alph{numpartie}~ ~#1}\hfill
        %[\;{\it bar\^eme indicatif : #2}\;]\\
        \\

}

\newenvironment{exercice}{\refstepcounter{exo} \vspace*{1em}\begin{thmbox}[S]{{\bf Exercice \arabic{exo}~:}}}{\end{thmbox}}

\newenvironment{solution}{\vspace*{1em}\begin{thmbox}[S]{{\bf Solution~:}}}{\end{thmbox}}

\newcommand{\IGNORE}[1]{}

\def\EnteteUE{
\noindent{\bf Université Grenoble Alpes} \hfill {\bf DLST}

\vspace{-8pt}

\noindent\hrulefill

\vspace{-2pt}

\noindent{\bf UE INF203} \hfill {\bf Année 2017-18}

\vspace*{.5cm}
}

\newcommand{\PiedDePageUE}[1]{
\fancypagestyle{monstyle}
{
\rhead{}
\lfoot{{\bf\small INF203 - 2016/2017}}
\cfoot{\thepage}
\rfoot{{\it #1}}
\renewcommand{\footrulewidth}{0.5pt}
\renewcommand{\headrulewidth}{0.0pt}
}
\pagestyle{monstyle}
}


\newcounter{question}
\newenvironment{question}[1][]{\refstepcounter{question}
   \textbf{[\bf \alph{question}] {\begin{emph}{#1}\end{emph}}}}{~$\blacksquare$~\\}

\newcommand{\unix}[1]{\hspace*{2cm}{\bf \tt #1}}
\newcommand{\fich}[1]{{\bf \em #1}}
\newcommand{\ecran}[1]{{\tt #1}}
\newcommand{\C}[1]{{\tt #1}}
\newcommand{\cmd}[1]{{\bf #1}}
\newcommand{\home}{\~{}}

\newenvironment{exocomp}{\vspace*{1em}\begin{thmbox}[M]{Exercice complémentaire :~}}{\end{thmbox}}


\begin{document}
\thispagestyle{empty}

\makeatletter
\lst@InstallKeywords k{attributes}{attributestyle}\slshape{attributestyle}{}ld
\makeatother

\definecolor{mygreen}{rgb}{0,0.3,0}
\definecolor{mygray}{rgb}{0.5,0.5,0.5}
\definecolor{mymauve}{rgb}{0.9,0,0.4}

\lstdefinestyle{customc}{
  belowcaptionskip=1\baselineskip,
  breaklines=true,
  frame=L,
  xleftmargin=\parindent,
  language=C,
  showstringspaces=false,
  basicstyle=\footnotesize\ttfamily,
  keywordstyle=\bfseries\color{green!40!black},
  commentstyle=\itshape\color{purple!40!black},
  identifierstyle=\color{blue},
  stringstyle=\color{orange},
}

\lstdefinestyle{transitionc}{
  belowcaptionskip=1\baselineskip,
  breaklines=true,
  frame=L,
  xleftmargin=\parindent,
  language=C,
  showstringspaces=false,
  basicstyle=\footnotesize\ttfamily,
  keywordstyle=\bfseries\color{green!40!black},
  commentstyle=\itshape\color{purple!40!black},
  identifierstyle=\color{blue},
  stringstyle=\color{orange},
  moreattributes={Q1, Q2, Q3, Q4},
  attributestyle = \bfseries\color{mymauve},
}

\lstdefinestyle{none}{
  belowcaptionskip=1\baselineskip,
  breaklines=true,
  frame=L,
  xleftmargin=\parindent,
  language=bash,
  showstringspaces=false,
  basicstyle=\footnotesize\ttfamily,
  keywordstyle=\bfseries\color{black},
  commentstyle=\itshape\color{black},
  identifierstyle=\color{black},
  stringstyle=\color{black},
}

\lstset{tabsize=3, style=customc,literate=
  {á}{{\'a}}1 {é}{{\'e}}1 {í}{{\'i}}1 {ó}{{\'o}}1 {ú}{{\'u}}1
  {Á}{{\'A}}1 {É}{{\'E}}1 {Í}{{\'I}}1 {Ó}{{\'O}}1 {Ú}{{\'U}}1
  {à}{{\`a}}1 {è}{{\`e}}1 {ì}{{\`i}}1 {ò}{{\`o}}1 {ù}{{\`u}}1
  {À}{{\`A}}1 {È}{{\'E}}1 {Ì}{{\`I}}1 {Ò}{{\`O}}1 {Ù}{{\`U}}1
  {ä}{{\"a}}1 {ë}{{\"e}}1 {ï}{{\"i}}1 {ö}{{\"o}}1 {ü}{{\"u}}1
  {Ä}{{\"A}}1 {Ë}{{\"E}}1 {Ï}{{\"I}}1 {Ö}{{\"O}}1 {Ü}{{\"U}}1
  {â}{{\^a}}1 {ê}{{\^e}}1 {î}{{\^i}}1 {ô}{{\^o}}1 {û}{{\^u}}1
  {Â}{{\^A}}1 {Ê}{{\^E}}1 {Î}{{\^I}}1 {Ô}{{\^O}}1 {Û}{{\^U}}1
  {œ}{{\oe}}1 {Œ}{{\OE}}1 {æ}{{\ae}}1 {Æ}{{\AE}}1 {ß}{{\ss}}1
  {ű}{{\H{u}}}1 {Ű}{{\H{U}}}1 {ő}{{\H{o}}}1 {Ő}{{\H{O}}}1
  {ç}{{\c c}}1 {Ç}{{\c C}}1 {ø}{{\o}}1 {å}{{\r a}}1 {Å}{{\r A}}1
  {€}{{\euro}}1 {£}{{\pounds}}1 {«}{{\guillemotleft}}1
  {»}{{\guillemotright}}1 {ñ}{{\~n}}1 {Ñ}{{\~N}}1 {¿}{{?`}}1
}

\EnteteUE

\begin{center}
  {\large {\bf Corrigé TP10}}
\end{center}

\vspace*{0.5cm}

\section{Encore des automates ...}

\begin{enumerate}[label=\textbf{[\alph*]}]
  \setlength\itemsep{1em}

\item Le code de la porte est \textbf{1234}. En effet, on peut voir
  qu'on arrive dans l'état initial \texttt{Q0} dans lequel tout
  chiffre différent de 1 ramène au même état. Puis, si on tape 1, on
  aboutit à l'état \texttt{Q1} dans lequel tout chiffre différent de 2
  ramène à l'état \texttt{Q0}, et ainsi de suite jusqu'au chiffre
  4. Seul la séquence d'entrée \textbf{1234} amène l'automate à
  produire la sortie \textbf{"clic"}.

\item Pour arrêter la simulation, il faut entrer \textbf{"-"}. En
  effet, si on observe la fonction \texttt{lire\_entree}, le programme
  demande à l'utilisateur d'entrer un caractère au clavier jusqu'à ce
  que le résultat soit un chiffre en 0 et 9, ou un tiret. Si c'est
  bien un chiffre qui est entré, la valeur du chiffre est renvoyée (
  \texttt{c - '0'} ), sinon -1. Et la fonction
  \texttt{simule\_automate} boucle tant que l'entrée est différente de
  -1.

\item Je n'avais absolument pas remarquée la fonction pour tester si
  un état est final. On peut donc utiliser le retour de la fonction
  \texttt{est\_final} dans la condition de notre \texttt{while} pour
  s'arrêter lorsque l'état final est atteint.

  \vspace{0.2cm}
  Voici une implémentation de Digicode\_V1 (Seulement le fichier
  \texttt{automate.c}, les autres restent inchangés) :

  \lstinputlisting{listing/Digicode_V1/automate.c}

  \newpage
  Voilà une implémentation possible pour la fonction
  \texttt{transition} et l'adaptation de la fonction
  \texttt{simule\_automate} correspondante.

  \vspace{0.2cm}
  On note que la sortie \textbf{"clic"} est affichée dans la fonction
  de transition. Ce n'est pas complètement satisfaisant et l'on
  pourrait plutôt séparer cette sortie dans une fonction à part.

  \lstinputlisting{listing/Digicode_V2/automate.c}

  \newpage
\item Voilà la représentation de la fonction de transition de notre
  digicode.

  \begin{center}
    \begin{tabular}{|l|c|c|c|c|c|}
      \hline
      & 1 & 2 & 3 & 4 & 0, 5, 6, 7, 8, 9 \\
      \hline
      \textbf{Q0} & Q1 & Q0 & Q0 & Q0 & Q0 \\
      \textbf{Q1} & Q1 & Q2 & Q0 & Q0 & Q0 \\
      \textbf{Q2} & Q1 & Q0 & Q3 & Q0 & Q0 \\
      \textbf{Q3} & Q1 & Q0 & Q0 & Q4 & Q0 \\
      \textbf{Q4} & Q1 & Q0 & Q0 & Q0 & Q0 \\
      \hline
    \end{tabular}
  \end{center}

  On note que dans cet représentation, les entrées 0, 5, 6, 7, 8, 9
  ont été factorisées car l'automate se comporte de la même manière
  dans tous ces cas. Mais ce ne sera pas toujours le cas, en
  particulier si l'on change le code ! Dans notre implémentation, nous
  choisirons donc de coder le tableau de transition en donnant une
  colone par chiffre. \\

  Voilà donc une implémentation du tableau des transitions dans
  Digicode\_V3, suivie de la fonction \texttt{simule\_automate}
  adaptée :

  \lstset{style=transitionc}
  \lstinputlisting{listing/Digicode_V3/automate.c}

\item Pour la sortie, on ne peut évidemment plus inclure la seule
  sortie \textbf{"clic"} dans le tableau de transition, comme on le
  faisait dans l'ammas de switchs. On a donc pris le parti, dans cette
  implémentation, d'intégrer la sortie dans la fonction
  \texttt{simule\_automate} à l'aide d'un petit \texttt{if}. On note
  bien que si l'automate avait plus sorties, il aurait sans doute
  fallu les traiter dans une fonction dédiée.

\item Nous voilà donc arrivé au moment fatidique tant attendu : il
  faut maintenant changer le code de la porte. Dans la version
  Digicode\_V3, le code est intégré dans le tableau de
  transition. Voilà un tableau de transition pour le code
  \textbf{2345} :

  \lstset{style=transitionc}
  \begin{lstlisting}
    etat transition [NB_ETATS][NB_ENTREES] = {
      /* entrees ->  0,  1,  2,  3,  4,  5,  6,  7,  8,  9 */
      /* Q0 */     { Q0, Q0, Q1, Q0, Q0, Q0, Q0, Q0, Q0, Q0 },
      /* Q1 */     { Q0, Q0, Q1, Q2, Q0, Q0, Q0, Q0, Q0, Q0 },
      /* Q2 */     { Q0, Q0, Q1, Q0, Q3, Q0, Q0, Q0, Q0, Q0 },
      /* Q3 */     { Q0, Q0, Q1, Q0, Q0, Q4, Q0, Q0, Q0, Q0 },
      /* Q4 */     { Q0, Q0, Q1, Q0, Q0, Q0, Q0, Q0, Q0, Q0 }
    };
  \end{lstlisting}

  \newpage

  Une autre implémentation pour le code \textbf{3874} :

  \lstset{style=transitionc}
  \begin{lstlisting}
    etat transition [NB_ETATS][NB_ENTREES] = {
      /* entrees ->  0,  1,  2,  3,  4,  5,  6,  7,  8,  9 */
      /* Q0 */     { Q0, Q0, Q0, Q1, Q0, Q0, Q0, Q0, Q0, Q0 },
      /* Q1 */     { Q0, Q0, Q0, Q1, Q0, Q0, Q0, Q0, Q2, Q0 },
      /* Q2 */     { Q0, Q0, Q0, Q1, Q0, Q0, Q0, Q3, Q0, Q0 },
      /* Q3 */     { Q0, Q0, Q0, Q1, Q4, Q0, Q0, Q0, Q0, Q0 },
      /* Q4 */     { Q0, Q0, Q0, Q1, Q0, Q0, Q0, Q0, Q0, Q0 }
    };
  \end{lstlisting}

  Le code peut bien sûr comporter plusieurs fois le même
  chiffre. Exemple avec un implémentation pour le code \textbf{1212} :

  \lstset{style=transitionc}
  \begin{lstlisting}
    etat transition [NB_ETATS][NB_ENTREES] = {
      /* entrees ->  0,  1,  2,  3,  4,  5,  6,  7,  8,  9 */
      /* Q0 */     { Q0, Q1, Q0, Q0, Q0, Q0, Q0, Q0, Q0, Q0 },
      /* Q1 */     { Q0, Q1, Q2, Q0, Q0, Q0, Q0, Q0, Q0, Q0 },
      /* Q2 */     { Q0, Q3, Q0, Q0, Q0, Q0, Q0, Q0, Q0, Q0 },
      /* Q3 */     { Q0, Q1, Q4, Q0, Q0, Q0, Q0, Q0, Q0, Q0 },
      /* Q4 */     { Q0, Q1, Q0, Q0, Q0, Q0, Q0, Q0, Q0, Q0 }
    };
  \end{lstlisting}

\item On remarque qu'on a modifié seulement le Digicode\_V3 dans cette
  correction. Pour modifier le code dans les version 1 et 2, il aurait
  fallu modifier les switchs, ce qui est très fastidieux. Dans la
  version \textbf{tabulaire}, on peut se concentrer sur la logique de
  l'automate indépendemment de l'implémentation. Cela en fait une
  version beaucoup plus maintenable, c'est à dire facile à modifier
  en limitant le risque d'introduction de nouveaux bugs.\\

  Bien entendu, quelle que soit la version choisie, seuls des tests
  rigoureux, et si possible automatiques, pourront nous donner
  confiance dans notre code.

\item Dans les exemples précédents de table de transition, les états
  différents de \texttt{Q0} sont coloriés en rouge. On voit donc bien
  que taper le premier chiffre du code doit toujours mener à l'état
  \texttt{Q1}, quel que soit l'état de départ. En effet, taper le bon
  premier chiffre est toujours potentiellement le début d'une bonne
  séquence.
  \vspace{0.2cm}

  En revanche, taper les chiffres suivants du code ne doit mener à
  l'état correspondant que si on a entré précédemment les bons
  chiffres. Ensuite, taper n'importe quel chiffre qui n'est pas dans
  la combinaison ou qui est précédé d'une combinaison erronnée doit
  ramener à l'état d'origine \texttt{Q0}.
  \vspace{0.2cm}

  Voilà pourquoi la table contient partout des \texttt{Q0}, sauf sur
  la colonne correspondant au premier chiffre du code, où elle
  contient des \texttt{Q1}, puis seulement un \texttt{Q2}, un
  \texttt{Q3} et un \texttt{Q4} aux transitions correspondant à la
  bonne séquence de code entrée.
  \vspace{2cm}

  \newpage

\item Dans la suite, on lit l'automate depuis un fichier de
  sauvegarde. Avant de se plonger dans le listing, voilà quelques
  explications sur les choix techniques.

  \begin{itemize}
  \item Le format choisi pour le fichier est le suivant :
\begin{verbatim}
1
2
3
4
Porte ouverte !
\end{verbatim}
    On reconnait les quatre chiffres du code ainsi que le message qui
    s'affiche lorsqu'on a entré la bonne combinaison. Certains
    d'entre-vous auront peut être ré-employé la methode du TP9, en
    sauvegardant l'intégralité de la table de transition et de la
    table de sortie. Ce n'est absolument pas nécessaire dans ce
    cas. En effet, connaissant le code, et le message de
    déverrouillage désiré, nous pourrons reconstituer ces deux
    tables. Ce fichier de sauvegarde contient donc l'intégratlité des
    données dont nous avons besoin.

  \item Par ailleurs, on a placé chaque chiffre sur une ligne pour
    éviter de lire un seul int comme \textbf{1234}, ce qui aurait
    nécessité quelques caluls pour extraire les chiffres
    individuellements. On a le choix du format, alors on en profite !

  \item Dans le fichier \texttt{automate.c}, vous ne trouverez pas la
    fonction \texttt{init\_par\_defaut} demandée dans le sujet du
    TP. C'est parce qu'elle est redondante avec l'initialisation de
    mon tableau de transition tel que je l'ai réalisé.

  \item On ajoute donc deux fonctions, \texttt{lecture\_automate} et
    \texttt{init\_automate} qui vont respectivement lire le fichier de
    sauvegarde et configurer la table de transition avec les valeurs
    obtenues. La fonction \texttt{lecture\_automate} devra maintenant
    être appelée dans le \texttt{main} avant d'utiliser
    l'automate. Donc elle doit être visible depuis l'extérieur du
    fichier \texttt{automate.c} ce qui justifie qu'elle soit déclarée
    dans le fichier \texttt{automate.h}

  \item Observée l'élégance avec laquelle j'ai employé une structure
    pour stocker le message de sortie !
  \end{itemize}
  \vspace{0.2cm}

  Sans plus attendre voici le code de \texttt{automate.c} :
  \vspace{0.2cm}

  \lstset{style=customc}
  \lstinputlisting{listing/Digicode_V4/automate.c}
  \vspace{0.2cm}

  Puis le code de \texttt{automate.h} :
  \vspace{0.2cm}

  \lstinputlisting{listing/Digicode_V4/automate.h}
  \vspace{0.2cm}

  Et enfin le code de \texttt{main.c}
  \vspace{0.2cm}

  \lstinputlisting{listing/Digicode_V4/main.c}
  \vspace{0.2cm}

  Sans oublier le fichier de sauvegarde :
  \vspace{0.2cm}

  \lstset{style=none}
  \lstinputlisting{listing/Digicode_V4/sauvegarde.sav}
  \vspace{0.2cm}


\end{enumerate}








%
%\item On a utilisé les commandes GDB suivantes :
%
%  \begin{center}
%    \begin{tabular}{|c|c|}
%      \hline
%      \texttt{'p'} & \texttt{'l'} & \\
%      \hline
%    \end{tabular}
%  \end{center}
%
%\item Pour tester les différentes fonctions que l'on va écrire
%  (\texttt{mon\_strlen}, \texttt{mon\_strcpy}, \texttt{mon\_strcat},
%  \texttt{mon\_strcmp}), pas le choix, il faut pour chacune d'entre
%  elles écrire un appel pour chacun des cas possibles. C'est un travail
%  long et fastidieux qu'il est pourtant indispensable de faire pour
%  s'assurer qu'un programme fonctionne correctement.
%
%\end{enumerate}
%
%\newpage
%
%\section{Implémentation possible pour \texttt{chaines.c}}
%
%\lstinputlisting{listing/chaines_func.c}
%
%\vspace{0.5cm}
%
%Avant de parcourir le \texttt{main} du fichier \texttt{chaines.c},
%notons plusieurs choses :
%
%\begin{itemize}
%\item \textbf{ATTENTION !} Toutes les fonctions ci-dessus font l'hypothèse que les
%  pointeurs qui leur sont passés en paramètre sont \textbf{non nuls}.
%\item \textbf{ATTENTION !} Toutes celles qui modifient une chaine de
%  caractère considèrent que la mémoire a déjà été allouée. Ex:
%  \texttt{mon\_strcat} considère que \texttt{destination} est un
%  tableau suffisament grand pour stocker le résultat de la
%  concaténation.
%\item On remarque que la fonction \texttt{mon\_strcpy} et les
%  suivantes n'utilisent pas \texttt{mon\_strlen}. Ce n'est pas par
%  masochisme, mais parce que cela permet d'éviter des parcours de
%  chaine inutiles ! Pour connaître la taille, il faut regarder chaque
%  caractère un par un jusqu'à la fin. C'est une opération coûteuse que
%  l'on doit éviter si l'on peut.
%\item Si vous lisez cette implémation consciencieusement, vous ne
%  manquerez pas de constater que \texttt{mon\_strcmp} est écrite un
%  peu légèrement ... En effet, il ne suffit pas qu'un caractère soit
%  situé avant un autre dans la table \texttt{ASCII} pour qu'il le
%  précède selon l'ordre lexicographique. Par exemple, selon la table
%  \texttt{ASCII}, le caractère \texttt{'Z'} est situé avant le
%  caractère \texttt{'a'}. Or la lettre z majuscule n'est pas avant la
%  lettre a minuscule dans l'alphabet. Pourtant, est-ce une mauvaise
%  implémentation ? Elle a le mérite d'être très simple. Par ailleurs,
%  elle est peut être suffisante selon le contexte, par exemple si l'on
%  sait que l'on ne traîte que des chaînes contenant des lettres
%  minuscules. Prenez donc garde à ne pas passer trop de temps à
%  perfectionner un programme qui résout déjà 99\% des cas !
%\end{itemize}
%
%\vspace{0.5cm}
%
%\lstinputlisting{listing/chaines_main.c}
%
%\vspace{0.5cm}
%
%Résultat de l'exécution de cette implémentation :
%
%\vspace{0.5cm}
%
%\begin{verbatim}
%corentin@gazelle:~ $ ./chaine
%un mot (pas plus grand que 50 caractères !) à mesurer ?
%maison
%longueur du mot ('maison') : 6
%
%mon_strcpy(resultat, "copie simple")          resultat = <copie simple>
%mon_strcpy(resultat, "")                      resultat = <>
%mon_strcat(resultat, "ajout1")                resultat = <ajout1>
%mon_strcat(resultat, " ajout2")               resultat = <ajout1 ajout2>
%mon_strcat(resultat, "")                      resultat = <ajout1 ajout2>
%
%mon_strcmp("", "")            = 0
%mon_strcmp("abc", "abc")      = 0
%mon_strcmp("abc", "xbc")      = -1
%mon_strcmp("abc", "abC")      = 1
%mon_strcmp("a", "abc")        = -1
%\end{verbatim}
%
%\section{Fabriquer des phrases}
%
%\begin{enumerate}[label=\textbf{[\alph*]},resume]
%  \setlength\itemsep{1em}
%
%\item On a bien sûr pensé, lors de l'écriture de \texttt{phrases.c}, à
%  ajouter des espaces entre les chaines. Cela donne donc :
%
%  \begin{lstlisting}
%        char Sujet[50]="la petite souris";
%        char Verbe[50]="mange";
%        char Compl[50]="le gros chat";
%        char Phrase[150];
%
%        mon_strcat(Phrase, Sujet);
%        mon_strcat(Phrase, " ");
%        mon_strcat(Phrase, Verbe);
%        mon_strcat(Phrase, " ");
%        mon_strcat(Phrase, Compl);
%  \end{lstlisting}
%
%\item Pour compter les mots dans la phrase, si on l'a construite
%  correctement, il suffit de compter les espaces. Sans oublier de
%  rajouter 1 à la fin, puisqu'il y a une espace entre chaque mot, mais
%  pas à la fin de la phrase. Implémentation complète de
%  \texttt{phrases.c} :
%
%  \vspace{0.5cm}
%
%  \lstinputlisting{listing/phrases.c}
%
%\end{enumerate}
%
%\section{Sacs de billes}
%
%
%\begin{enumerate}[label=\textbf{[\alph*]},resume]
%  \setlength\itemsep{1em}
%
%\item Dans la fonction \texttt{lire\_joueur}, on veut mettre à jour la
%  variable contenant les informations d'un joueur avec ce que
%  l'utilisateur aura tapé au clavier. On doit donc connaître l'adresse
%  où est effectivement stockée cette variable. Or, dans le prototype
%  de fonction : \texttt{lire\_joueur(joueur j)}, le paramètre
%  \texttt{j} n'est pas un pointeur. C'est donc un passage par
%  \textbf{copie}. Si l'on modifie \texttt{j} dans cette fonction,
%  c'est la copie fabriquée pour la fonction qui sera modifiée et non
%  pas la variable originale.
%
%\item Une implémentation possible de billes :
%
%\vspace{0.5cm}
%
%  \lstinputlisting{listing/billes.c}
%
%\end{enumerate}
%
%\newpage
%
%\section{La command \texttt{diff}}
%
%On considère les fichiers \texttt{d1.c}, \texttt{d2.c} et \texttt{d3.c}
% contenant respectivement :
%
%  \lstinputlisting{listing/d1.c}
%  \lstinputlisting{listing/d2.c}
%  \lstinputlisting{listing/d3.c}
%
%\vspace{0.5cm}
%
%À première vue, ils semblent identiques. Voici le résultat de
%plusieurs commandes \texttt{diff}.
%
%\begin{Verbatim}[xleftmargin=2em]
%corentin@gazelle:listing $ diff d1.c d2.c
%corentin@gazelle:listing $
%\end{Verbatim}
%
%On voit que le diff de \texttt{d1.c} et \texttt{d2.c} se termine
%sans rien imprimer sur l'écran. C'est donc que les deux fichiers sont
%identiques.
%
%\begin{Verbatim}[xleftmargin=2em]
%corentin@gazelle:listing $ diff d1.c d3.c
%4c4
%<       printf("Hello, world!");
%---
%>       printf("Hello, world!\n");
%corentin@gazelle:listing $
%\end{Verbatim}
%
%En revanche, le diff de \texttt{d1.c} et \texttt{d3.c} donne la
%liste des différences entre les deux fichiers. Plusieurs
%explications~:
%
%\begin{description}[leftmargin=!,labelwidth=\widthof{\bfseries gauche/droite}]
%\item[gauche/droite] Ici, on appelle \textit{fichier de gauche}
%\texttt{d1.c} car c'est le premier paramètre donné à \texttt{diff}. Il
%est donc situé à gauche sur la ligne de commande. Par suite,
%\texttt{d3.c} est le fichier de droite.
%\item[4c4] \textbf{4} $\rightarrow$ ligne 4 de \texttt{d1.c} \\
%  \textbf{c} $\rightarrow$ a changé \\
%  \textbf{4} $\rightarrow$ correspond désormais à ligne 4 de \texttt{d3.c}
%\item[\texttt{<}] Indique que la ligne qui suit se trouve dans le fichier de
%  gauche
%\item[\texttt{>}] Indique que la ligne qui suit se trouve dans le gichier de
%  droite
%\item[\texttt{------}] Sépare les lignes situés dans des fichiers différents
%\end{description}
%
\end{document}
