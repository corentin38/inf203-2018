\documentclass[10pt]{article}

\usepackage{graphics}
\usepackage{dirtree}
\usepackage{paracol}
\usepackage{epigraph}
\usepackage{enumitem}
\usepackage{xcolor}
\usepackage{fancyvrb}
\usepackage{calc}

\usepackage{pgf}
\usepackage{tikz}
\usetikzlibrary{arrows,automata,positioning}
\usepackage{epigraph}
\usepackage{eurosym}

\setlength{\textwidth}{17cm}
\setlength{\oddsidemargin}{-1cm}
\setlength{\evensidemargin}{-1cm}
\setlength{\textheight}{26cm}
\setlength{\parindent}{0pt}
\setlength{\parskip}{0.3ex}

\usepackage{fancyhdr}
\usepackage{epsfig}
\usepackage[utf8]{inputenc}
\usepackage[french]{babel}
\usepackage[T1]{fontenc}
%\usepackage{verbatim}
\usepackage{graphics}
\usepackage{amsmath,amsfonts,amssymb}
\usepackage{listings}
\usepackage{thmbox}
\usepackage{comment}

\oddsidemargin=0cm
\evensidemargin=0cm
\textwidth=17cm
\textheight=25cm
\topmargin=0mm
\hoffset=-6mm
\voffset=-25mm
%\headsep=30pt

\newcounter{numpartie}
\newcounter{exo}
\newcommand{\partie}[2]
{
        \refstepcounter{numpartie}
        \setcounter{exo}{0}
        \vspace{3mm}

        %\noindent{\large\bf Partie \Alph{numpartie}~-~#1}\hfill
        \noindent{\large\bf Partie \Alph{numpartie}~ ~#1}\hfill
        %[\;{\it bar\^eme indicatif : #2}\;]\\
        \\

}

\newenvironment{exercice}{\refstepcounter{exo} \vspace*{1em}\begin{thmbox}[S]{{\bf Exercice \arabic{exo}~:}}}{\end{thmbox}}

\newenvironment{solution}{\vspace*{1em}\begin{thmbox}[S]{{\bf Solution~:}}}{\end{thmbox}}

\newcommand{\IGNORE}[1]{}

\def\EnteteUE{
\noindent{\bf Université Grenoble Alpes} \hfill {\bf DLST}

\vspace{-8pt}

\noindent\hrulefill

\vspace{-2pt}

\noindent{\bf UE INF203} \hfill {\bf Année 2016-17}

\vspace*{.5cm}
}

\newcommand{\PiedDePageUE}[1]{
\fancypagestyle{monstyle}
{
\rhead{}
\lfoot{{\bf\small INF203 - 2016/2017}}
\cfoot{\thepage}
\rfoot{{\it #1}}
\renewcommand{\footrulewidth}{0.5pt}
\renewcommand{\headrulewidth}{0.0pt}
}
\pagestyle{monstyle}
}


\newcounter{question}
\newenvironment{question}[1][]{\refstepcounter{question}
   \textbf{[\bf \alph{question}] {\begin{emph}{#1}\end{emph}}}}{~$\blacksquare$~\\}

\newcommand{\unix}[1]{\hspace*{2cm}{\bf \tt #1}}
\newcommand{\fich}[1]{{\bf \em #1}}
\newcommand{\ecran}[1]{{\tt #1}}
\newcommand{\C}[1]{{\tt #1}}
\newcommand{\cmd}[1]{{\bf #1}}
\newcommand{\home}{\~{}}

\newenvironment{exocomp}{\vspace*{1em}\begin{thmbox}[M]{Exercice complémentaire :~}}{\end{thmbox}}


\begin{document}
\thispagestyle{empty}

\makeatletter
\lst@InstallKeywords k{attributes}{attributestyle}\slshape{attributestyle}{}ld
\makeatother

\definecolor{mygreen}{rgb}{0,0.3,0}
\definecolor{mygray}{rgb}{0.5,0.5,0.5}
\definecolor{mymauve}{rgb}{0.9,0,0.4}

\lstdefinestyle{customc}{
  belowcaptionskip=1\baselineskip,
  breaklines=true,
  frame=L,
  xleftmargin=\parindent,
  language=C,
  showstringspaces=false,
  basicstyle=\footnotesize\ttfamily,
  keywordstyle=\bfseries\color{green!40!black},
  commentstyle=\itshape\color{purple!40!black},
  identifierstyle=\color{blue},
  stringstyle=\color{orange},
}

\lstdefinestyle{none}{
  belowcaptionskip=1\baselineskip,
  breaklines=true,
  frame=L,
  xleftmargin=\parindent,
  language=bash,
  showstringspaces=false,
  basicstyle=\footnotesize\ttfamily,
  keywordstyle=\bfseries\color{black},
  commentstyle=\itshape\color{black},
  identifierstyle=\color{black},
  stringstyle=\color{black},
}

\lstset{tabsize=3, style=customc,literate=
  {á}{{\'a}}1 {é}{{\'e}}1 {í}{{\'i}}1 {ó}{{\'o}}1 {ú}{{\'u}}1
  {Á}{{\'A}}1 {É}{{\'E}}1 {Í}{{\'I}}1 {Ó}{{\'O}}1 {Ú}{{\'U}}1
  {à}{{\`a}}1 {è}{{\`e}}1 {ì}{{\`i}}1 {ò}{{\`o}}1 {ù}{{\`u}}1
  {À}{{\`A}}1 {È}{{\'E}}1 {Ì}{{\`I}}1 {Ò}{{\`O}}1 {Ù}{{\`U}}1
  {ä}{{\"a}}1 {ë}{{\"e}}1 {ï}{{\"i}}1 {ö}{{\"o}}1 {ü}{{\"u}}1
  {Ä}{{\"A}}1 {Ë}{{\"E}}1 {Ï}{{\"I}}1 {Ö}{{\"O}}1 {Ü}{{\"U}}1
  {â}{{\^a}}1 {ê}{{\^e}}1 {î}{{\^i}}1 {ô}{{\^o}}1 {û}{{\^u}}1
  {Â}{{\^A}}1 {Ê}{{\^E}}1 {Î}{{\^I}}1 {Ô}{{\^O}}1 {Û}{{\^U}}1
  {œ}{{\oe}}1 {Œ}{{\OE}}1 {æ}{{\ae}}1 {Æ}{{\AE}}1 {ß}{{\ss}}1
  {ű}{{\H{u}}}1 {Ű}{{\H{U}}}1 {ő}{{\H{o}}}1 {Ő}{{\H{O}}}1
  {ç}{{\c c}}1 {Ç}{{\c C}}1 {ø}{{\o}}1 {å}{{\r a}}1 {Å}{{\r A}}1
  {€}{{\euro}}1 {£}{{\pounds}}1 {«}{{\guillemotleft}}1
  {»}{{\guillemotright}}1 {ñ}{{\~n}}1 {Ñ}{{\~N}}1 {¿}{{?`}}1
}

\EnteteUE

\begin{center}
  {\large {\bf Corrigé TP9}}
\end{center}

\vspace*{0.5cm}

\begin{center}
  {\large {\bf Automates}}
\end{center}

\begin{enumerate}[label=\textbf{[\alph*]}]
  \setlength\itemsep{1em}

\item En lisant la fonction d'initialisation de l'automate
  \texttt{init\_par\_defaut}, on voit que pour chaque état, on
  paramètre une transition retournant au même état, quelle que soit
  l'entrée. Au niveau de l'instruction :

  \lstset{style=customc}
  \begin{lstlisting}
    A->transitions[i][j]=i;
  \end{lstlisting}

  En effet, la variable \texttt{A} de type \texttt{automate} contient
  la table de transition qui à chaque couple état/entrée
  (ligne/colonne) fait correspondre un nouvel état. Dans l'instruction
  ci dessus, l'indice \texttt{i} parcourt tous les numéros d'état et
  l'indice \texttt{j} tous les numéros d'entrée. Et dans la case à la
  ligne \texttt{i} et à la colonne \texttt{j}, on place le numéro
  d'état \texttt{i} correspondant donc à l'état d'où l'on part.

\item En lisant la définition du type \texttt{automate} :

  \lstset{style=customc}
  \begin{lstlisting}
    typedef struct {
      int nb_etats;
      int etat_initial;
      int etats_finals[NB_MAX_ETATS];
      int transitions[NB_MAX_ETATS][NB_MAX_ENTREES];
      char sortie[NB_MAX_ETATS][NB_MAX_ENTREES][LG_MAX_SORTIE];
    } automate;
  \end{lstlisting}

  On voit que la table de transition contient \texttt{NB\_MAX\_ETATS}
  lignes et \texttt{NB\_MAX\_ENTREES} colonnes. Ces deux constantes
  valent 128, ce qui correspond au nombre de caractères de la table
  ASCII. L'ensemble d'entrée de l'automate pourra donc potentiellement
  être \textbf{l'ensemble des caractères ASCII}.
  \vspace{0.2cm}

  Mais en observant la fonction \texttt{init\_mon\_automate}, on voit
  que seuls 2 états sont utilisés ainsi que 3 entrées : les caractères
  '2', 'r' et 'c'. Toutes les transitions utilisant d'autres
  caractères mènent à une sortie \texttt{"entree\_invalide"}. La
  conclusion est que notre implémentation permet de gérer n'importe
  quel automate acceptant des caractères ASCII, et que l'automate qui
  actuellement paramètré a l'ensemble d'entrée : \texttt{\{2,r,c\}}.
  \vspace{0.2cm}

  Cet ensemble d'entrée correspond évidemment aux actions :
  \textit{"mettre une pièce de 20 centimes"}, \textit{"appuyer sur le
    bouton café"} et \textit{"appuyer sur le bouton rendre monnaie"} que l'on
  pourrait effectuer sur une machine à café.

\item Lorsque l'on saisi des caractères non prévus, comme expliqué
  ci-dessus, on reste dans le même état et le message
  "entree\_invalide" s'affiche.

\item Voici une implémentation possible pour Café1.
  \vspace{0.2cm}

  On commence bien sûr par rédiger rapidement un Makefile (puisqu'on a
  l'habitude et qu'on est très à l'aise avec les Makefiles).

  \lstset{style=none}
  \begin{lstlisting}
    automate: automate.o main.o
        clang automate.o main.o -o automate

    automate.o: automate.c automate.h
            clang -c automate.c

    main.o: main.c automate.h
            clang -c main.c
  \end{lstlisting}
  \vspace{0.2cm}

  \newpage
  On complète ensuite la fonction \texttt{simule\_automate} comme
  suit~:

  \lstset{style=customc}
  \lstinputlisting{listing/Cafe1/extrait.c}
  \vspace{0.2cm}

\item Nous passons maintenant à l'implémentation de Cafe2, qui va lire
  l'automate depuis un fichier avant de le simuler. On commence par
  écrire la fonction \texttt{lecture\_automate} qui interprète le
  fichier \texttt{Mon\_automate.auto}.
  \vspace{0.2cm}

  Le format du fichier est bien décrit dans le sujet du TP. Vous savez
  manipuler la fonction \texttt{fscanf} pour lire tous les entiers et
  les caractères requis, voilà donc sans plus attendre une
  implémentation possible :

  \lstset{style=customc}
  \lstinputlisting{listing/Cafe2/extrait.c}
  \vspace{0.2cm}

  On note que l'on met une chaîne vide dans toutes les sorties
  correspondant à une transition lue, puis qu'on la reconfigure
  lorsqu'on lit la section des sorties. Ce n'est pas une erreur : on
  n'est en effet pas obligé de définir une sortie pour toutes les
  transitions que l'on a écrites. Une transition sans sortie doit donc
  logiquement afficher une chaîne vide. Sans cette assignation, la
  sortie d'une telle transition serait celle par defaut, c'est à dire
  \textbf{entree\_invalide}.

\item Voir listing ci-dessus pour \texttt{lecture\_automate}. Voici
  également la fonction main :

  \lstset{style=customc}
  \lstinputlisting{listing/Cafe2/main.c}
  \vspace{0.2cm}

  Et le header modifié :

  \lstset{style=customc}
  \lstinputlisting{listing/Cafe2/automate.h}
  \vspace{0.2cm}
  \newpage

\item Pour connaître la taille d'un type, on peut utiliser
  l'instruction \texttt{sizeof}. On passe en paramètre à cette
  dernière le nom du type que l'on souhaite, comme s'il s'agissait
  d'une fonction. Le résultat retourné est la taille qu'occuperait une
  variable du type choisi en octet. \texttt{sizeof(int)} par
  exemple, renvoit 4 sur mon ordinateur.
  \vspace{0.2cm}

  Attention, cela ne marche que pour les types statiques ! Il ne faut
  \textbf{surtout pas} avoir l'idée saugrenue de s'en servir pour
  chercher le nombre d'élément d'un tableau ou la longueur d'une
  chaîne de caractère !
  \vspace{0.2cm}

  Pour connaître la taille du type automate, on rajoute donc dans le
  main la ligne :

\begin{lstlisting}
        printf ("taille automate : %d\n", sizeof(automate));
\end{lstlisting}
  \vspace{0.2cm}

  Et l'on voit que la taille de l'automate est de 2163208 octets. Soit
  environs \textbf{2 méga octets} !
  \vspace{0.2cm}

  On comprend vite pourquoi : pour être générique, notre
  implémentation permet de définir 128 états et 128 entrées. Il y a
  donc 128 * 128 = 16384 transitions possibles. C'est le nombre de
  case de notre tableau de transition, ce qui en fait déjà un grand
  tableau ...
  \vspace{0.2cm}

  Pire ! Notre implémentation permet de stocker un message de sortie
  de 128 caractères maximum pour \textbf{chaque} transition. Le
  tableau de sortie contient donc 128 * 128 * 128 = 2097152 octets,
  soit 2Mo à lui tout seul. C'est là que la taille de l'automate
  explose véritablement.
  \vspace{0.2cm}

  On comprends donc que des choix qui paraissent raisonables aux
  premiers abords (128 états possible, 128 entrées possibles et 128
  caractères pour une sortie, ces choix pris individuellement semblent
  tout à fait acceptables) peuvent en fait conduire très rapidement à
  une forte consommation de mémoire. Prudence, donc ! D'autant que
  dans notre cas, lorsque l'on paramètre l'automate café1 dans notre
  programme, seuls 2 états, 3 entrées et 4 sorties sont réellement
  utilisées. Nos tableaux de transition et de sortie sont donc
  quasiment vides ! En conclusion, notre implémentation d'automate a
  le mérite d'être simple à comprendre et à utiliser, mais elle n'est
  \textbf{vraiment pas} optimale !

\item Voilà un nouvel automate pour notre machine à café 2.0.

  \begin{center}
    \begin{tikzpicture}[->,>=stealth',shorten >=1pt,auto,node distance=4.8cm, semithick]
      \tikzstyle{every state}=[draw=black,text=black]

      \node[initial,state,initial where=left]  (A)                    {$q_0$};
      \node[state]                             (B) [above right of=A] {20c  };
      \node[state]                             (C) [below right of=A] {10c  };
      \node[state]                             (D) [above right of=C] {30c  };

      \path (A) edge [bend left]  node {2}     (B)
                edge [loop above] node {c,t,r} (A)
                edge [bend right] node {c,t,r} (C)
            (B) edge [bend left]  node {r}     (A)
                edge [bend left]  node {1,2}   (D)
                edge [loop above] node {c,t}   (B)
            (C) edge [bend right] node {r}     (A)
                edge [pos=0.4]    node {1}     (B)
                edge [loop below] node {c,t}   (C)
                edge [bend right] node {2}     (D)
            (D) edge [pos=0.25]   node {c,t,r} (A)
                edge [loop right] node {1,2}   (D);
    \end{tikzpicture}
  \end{center}

  On a omis les message de sortie sur le schéma pour ne pas le rendre
  illisible. On remarque qu'il faut deux états intermédiaires
  supplémentaires pour gérer les pièces de 10 et 20 centimes. Sinon,
  l'automate ressemble fortement à la version précédente.
  \newpage

  Voilà son fichier auto pour simuler l'automate avec notre
  programme~:

  \lstset{style=none}
  \lstinputlisting{listing/Cafe2/Autre_automate.auto}
  \vspace{0.2cm}

\item Pas de spoil, on ne donnera pas ici la solution du mystère !
  Simplement l'état final de l'automate est l'état 2. On peut modifier
  la contidition du while dans la fonction \texttt{simule\_automate}
  pour sortir de la simulation lorsqu'on arrive à un état final. Voilà
  une façon de faire :

  \lstset{style=customc}
  \lstinputlisting{listing/Cafe2/extrait2.c}
  \vspace{0.2cm}

  On se souvient que le tableau \texttt{A->etat\_finals} contient une
  case pour chaque état. Si l'état 2 est final, alors la case 2 du
  tableau contient 1. Sinon, elle contient 0. Comme vous le savez, en
  C, 0 est évalué comme FAUX et 1 comme VRAI. Donc
  \texttt{A->etat\_finals[i]} est VRAI si l'état \texttt{i} est final,
  et FAUX sinon. Dans la fonction, on simule tant que l'état courant
  n'est pas un état final. Donc tant que
  \texttt{A->etat\_finals[etat\_courant]} est FAUX.

\end{enumerate}
\end{document}
